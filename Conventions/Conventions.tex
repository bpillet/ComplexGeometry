\documentclass[12pt,makeidx]{amsart}
\usepackage[couleur,draft]{amsdip}
\usepackage[utf8x]{inputenc}
\usepackage[T1]{fontenc}
\geometry{papersize={150mm,243mm},left=2em,right=2em,top=3em,bottom=3em}
%\geometry{a4paper,left=2em,right=2em,top=3em,bottom=3em}
%\setlength{\columnsep}{3.5em}
\pagenumbering{gobble}
\usepackage[charter]{mathdesign}
%\usepackage[normalmargins]{savetrees}
\sloppy
\pagestyle{empty}
\renewcommand*{\thefootnote}{\fnsymbol{footnote}}

\makeindex
\begin{document}

\section{Notations universelles}
\medskip
\begin{center}
\fbox{$\tau = 2\pi$}
\end{center}
\medskip

On fixe une identification de $A \otimes A \cong \left(A \wedge A\right) \oplus \left(A \odot A\right)$ en posant
\[
a \otimes b = a \wedge b + a \odot b
\]


\section{Volume et métriques}
\subsection{Cas riemannien}
Soit $(M,g)$ une variété riemannienne orienté. Alors dans \textit{n'importe quelle carte} (orientée) $(x^i)_i$
on peut écrire
\begin{equation}\label{MetriqueRiemann}\index{MetriqueRiemann@Métrique riemannienne}
g_x = \sum_{i,j} g_{ij}(x) \dd x^i \otimes \dd x^j
\end{equation}
Le tenseur $g_{ij}$ est
\begin{itemize}
\item Symétrique : $g_{ij} = g_{ji}$
\item Positif : $\sum g_{ij}x^ix^j \geq 0$
\item Défini : $\left(\sum g_{ij}x^ix^j = 0 \right)\Rightarrow \left(x = 0\right)$
\end{itemize}

On définit la forme de volume riemannienne 
\begin{equation}\label{VolumeRiemann}\index{VolumeRiemann@Volume riemannien}
\dd Vol = \sqrt{\det(g)} \dd x^1 \wedge \cdots \wedge \dd x^n
\end{equation}
Elle ne dépend pas du choix de la carte !

On peut remarquer que \begin{equation}
\dd Vol = E^*_1 \wedge \cdots  \wedge E^*_n
\end{equation}
où $(E_i)_i$ base orthonormée orientée de $TM$ et $(E_i^*)$ la base duale.

\subsection{Cas hermitien}
Soit $X$ une variété complexe. Alors une métrique hermitienne $h$ sur $X$ peut s'écrire dans \textit{n'importe quelle carte} holomorphe $(z^i)_i$ comme
\begin{equation}\label{MetriqueHermit}\index{MetriqueHermit@Métrique hermitienne}
h_x = \sum_{i,j} h_{ij}(z) \dd z^i \otimes \dd \bar{z}^j
\end{equation}
Rq : Une métrique hermitienne est antiholomorphe à droite

On lui associe~:
\begin{itemize}
\item $\omega = - Im(h)$ la forme fondamentale \cite{Demailly}
\item $g = Re(h)$ la métrique riemannienne sous-jacente
\end{itemize}
Dès lors
\begin{equation}
h = g - i \omega
\end{equation}
d'où
\begin{equation}
g(I\_,\_) = \omega
\end{equation}
et
\begin{equation}
\omega(I\_,\_) = -g
\end{equation}

On pose $\dd V_\omega = \omega^n/n!$ forme volume hermitienne \cite{Demailly}

On a la relation
\begin{equation}\label{VolumeRiemannHermit}
\dd V_{\omega} = 2^n \dd Vol
\end{equation}

\subsection{Sur $\Pro^1$}
\[
\omega_{FS} = i\ddbarre \log (1+\zeta\bar\zeta) = \dfrac{i\dd \zeta \wedge \dd \bar\zeta}{\left(1+\zeta\bar\zeta\right)^2}
\]
En posant $\zeta = re^{i\theta}$ on a 
\[
i\dd \zeta \wedge \dd \bar\zeta = 2r \dd r \wedge \dd \theta
\]
et donc on en déduit que
\[
\int_{\Pro^1} \omega_{FS} = \int_0^\tau \int_0^\infty \dfrac{2r \dd r \dd \theta}{(1+r^2)^2}
= \tau
\]
Et $Vol(\Pro^1) = \demi\int_{\Pro^1} \omega = \pi$ : La sphère de rayon $\demi$.

\subsubsection{Courbure} Une section locale du tangent de $\Pro^1$ est un champ de vecteur local. Prenons $X = \partial_\zeta$ alors sa norme relativement à la métrique $\omega_{FS}$ est 
\[
\vert \partial_\zeta \vert^2 = \dfrac{1}{\left(1+\zeta\bar\zeta\right)^2}
\]
D'où la courbure de $\Pro^1$ est donnée dans cette carte par
\[
\Theta(T\Pro^1) = -i\ddbarre \log \left(\dfrac{1}{\left(1+\zeta\bar\zeta\right)^2}\right) = 2 \omega_{FS}
\]

\appendix
\printindex
\end{document}
