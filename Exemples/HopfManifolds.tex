\documentclass[a4paper]{article}
\usepackage[thm]{../dipneuste}

\begin{document}
\section{La dimension $1$}
\section{La dimension $n \geq 2$}
Soit $\lambda = (\lambda_1,\cdots, \lambda_n) \in \C^n$ avec pour tout $k$, $0<|\lambda_k|<1$
Alors
\begin{equation}
\left(
\begin{array}{cccc}
\lambda_1 & 0 & \cdots & 0 \\ 
0 & \lambda_2 & \ddots  & \vdots \\ 
\vdots & \ddots & \ddots & 0 \\ 
0 & \cdots & 0 & \lambda_n
\end{array} 
\right)
\end{equation}
Agit sur $\C^n \setminus 0$ par multiplication à gauche. Les orbites sont discrètes.

Ainsi le quotient
\begin{equation}
M_\lambda^n := \left(\C^n \setminus 0\right) \big/ \lambda
\end{equation}
est une variété complexe.

De plus comme $\C^n \setminus 0$ est simplement connexe (car $n \geq 2$), c'est le revêtement universel de $M_\lambda^n$. Les fibres de $\pi : \C^n \setminus 0 \to M_\lambda^n$ sont isomorphes à $\Z$.

\subsection{Fonctions méromorphes}
Si $F : M_\lambda^n \dashrightarrow \Pro^1$ est une fonction méromorphe, alors $F\circ \pi : \C^n \setminus 0 \dashrightarrow \Pro^1$ est aussi une fonction méromorphe $\lambda$-périodique au sens où
\begin{equation}
\forall z \in \C^n \setminus 0\ ,\quad (F\circ \pi)(\lambda z) = (F\circ \pi)(z)
\end{equation}

\begin{thm}[Hartogs-Levi]
Soit $n \geq 2$ et $\Delta \in \C^n$ un polydisque ouvert contenant $0$. Soit $f : \Delta \setminus 0 \to \C$ une fonction holomorphe, alors $f$ s'étend de manière unique en une fonction holomorphe sur $\Delta$.
\end{thm}
\begin{thm}[Hartogs-Levi, version méromorphe \label{HartogsMero}]
Soit $n \geq 2$ et $\Delta \in \C^n$ un polydisque ouvert contenant $0$. Soit $f : \Delta \setminus 0 \dashrightarrow \Pro^1$ une fonction méromorphe, alors $f$ s'étend de manière unique en une fonction méromorphe sur $\Delta$.
\end{thm}

Le théorème \ref{HartogsMero} nous dit donc que $G = F \circ \pi$ s'étend en une fonction méromorphe sur $\C^n$. Elle vérifie toujours
\begin{equation}\label{periodique}
\forall z \in \C^n \ , \quad G(\lambda z) = G(z)
\end{equation}

\subsection{Étude locale}
Soit $\Delta$ un polydisque voisin de $0$ suffisamment petit pour que $G$ s'écrive $G_1/G_2$ avec $G_1, G_2$ holomorphes sur $\Delta$.
De plus $\lambda : \Delta \to \lambda\Delta \subseteq \Delta$ car tous les $\lambda_i$ sont de module $<1$.

Alors l'équation \eqref{periodique} nous donne
\[
\dfrac{G_1 \circ \lambda}{G_2 \circ \lambda} = \dfrac{G_1}{G_2}
\]
En particulier, $G_1 \circ \lambda$ s'annule sur les $0$ de $G_1$ et de même $G_2 \circ \lambda$ s'annule sur les pôles de $G_2$.

\begin{thm}[Lemme de préparation de Weierstrass]
Soit $f$ holomorphe sur $U \subseteq \C^n$ tel que $0 \in U$. Supposons que $f$ s'annule en $0$, alors, quitte à restreindre $U$
\begin{itemize}
\item il existe $g$ holomorphe sur $U$ qui ne s'annule pas (inversible).
\item il existe $p = f_0 + f_1z + $
\end{itemize}
...
\end{thm}

On trouve $G_i \circ \lambda = U_iG_i$, avec $U_i$ holomorphe inversible. Ainsi on a
\[
\dfrac{U_1 G_1}{U_2 G_2} = \dfrac{G_1}{G_2}
\]
d'où $U_1 = U_2$. Que l'on notre $U$.




\subsection*{Cas de la dimension $2$}
On pose $z = (z_1,z_2)$ et $\lambda = (\lambda_1,\lambda_2)$

Supposons $G$ non constante (on cherche une contradiction). De plus si $G(0) \neq 0$ alors $G - G(0)$ vérifie toujours les même propriétés ($\lambda$-périodicité, méromorphe sur $\C^n$), on peut donc se ramener au cas où $G(0) = 0$.

En particulier la valuation de $G_1$ est strictement plus élevée que celle de $G_2$ car $G$ s'annule en $0$. 

Posons $v$ la valuation de $G_1$, on peut écrire donc
\[
G_1(z) = \sum_{k=0}^v a_k z_1^kz_2^{v-k} + o(|z|^v)
\]
La relation $G_1 \circ \lambda = UG_1$ entraîne que
\[
\sum_{k=0}^v a_k \lambda_1^k\lambda_2^{v-k}z_1^kz_2^{v-k} = \sum_{k=0}^v U(0)a_k z_1^kz_2^{v-k}
\]
Ainsi
\begin{align*}
a_0\lambda_1^0\lambda_2^{v} &= U(0) a_0\\
a_1\lambda_1^1\lambda_2^{v-1} &= U(0) a_1\\
&\cdots\\
a_v \lambda_1^v \lambda_2^0 &= U(0) a_v
\end{align*}
Supposons dans un premier temps que deux des $a_i$ soient non nuls : $a_s$ et $a_t$ pour $s<t$. Alors
\begin{equation}
\lambda_1^s \lambda_2^{v-s} = U(0) = \lambda_1^t\lambda_2^{v-t}
\end{equation}
et donc
\[
\lambda_1^{t-s}\lambda_2^{s-t} = 1
\]

\textit{On fait ici l'hypothèse "générique" (presque toujours satisfaite), que les $\lambda_i$ sont algébriquement indépendants, c'est-à-dire~:}
\[
\forall (l_1,\cdots, l_n) \in \Z^n\ ,\qquad \prod_{i =1}^n \lambda_i^{l_i} = 1 \ \Rightarrow\ l_1=l_2=\cdots =l_n = 0
\]

Donc $s=t$ ainsi il est impossible que $2$ des coefficients $a_i$ soient non nuls. Nécessairement $G_1$ s'écrit 
\[
a z_1^s z_2^{v-s} + o(|z|^v)
\] au voisinage de $0$.

Le même raisonnement est valable pour $G_2$ de valuation $w < v$ et donc s'écrit
\[
b z_1^{s'} z_2^{w-s'} +  o(|z|^w)
\] au voisinage de $0$.

Donc
\[
G(z) = \dfrac{a z_1^s z_2^{v-s} + o(|z|^v)}{b z_1^{s'} z_2^{w-s'} +  o(|z|^w)}
\]%Q_{w+1}(\lambda z) +  \cdots + Q_v(\lambda z) + o(|z|^v)
Mais $G(\lambda z) = G(z)$ entraîne
\[
G(z) = \dfrac{\lambda_1^{s}\lambda_2^{v-s} a z_1^s z_2^{v-s} + o(|z|^v)}{\lambda_1^{s'}\lambda_2^{w-s'}b z_1^{s'} z_2^{w-s'} +  o(|z|^w)}
\]
En identifiant on trouve
\[
\left(\lambda_1^{s}\lambda_2^{v-s}a z_1^s z_2^{v-s} + o(|z|^v)\right)\left(b z_1^{s'} z_2^{w-s'} + o(|z|^w)\right) =\left(a z_1^s z_2^{v-s} + o(|z|^v)\right)\left( \lambda_1^{s'}\lambda_2^{w-s'}b z_1^{s'} z_2^{w-s'} + o(|z|^w)\right)
\]
et dont le terme de bidegré $(s+s',w+v-s-s')$ nous donne après simplification par $a$ et $b$
\[
\lambda_1^{s}\lambda_2^{v-s} = \lambda_1^{s'}\lambda_2^{w-s'}
\]
Mais par notre hypothèse d'indépendance algébrique, nécessairement $s=s'$ et $v-s=w-s'$ donc $v=w$ ce qui est une contradiction avec l'annulation de $G$ en $0$.

\end{document}