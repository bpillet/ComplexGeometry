\documentclass[12pt,titlepage]{article}

\usepackage{amsmath}
\usepackage{amsfonts}
\usepackage{amssymb}
\usepackage{amsthm}
\usepackage{mathtools}
\usepackage{graphicx}
\usepackage{color}
\usepackage{ucs}
\usepackage[utf8x]{inputenc}
\usepackage{xparse}
\usepackage{hyperref}

%----Macros----------
%
% Unresolved issues:
%
%  \righttoleftarrow
%  \lefttorightarrow
%
%  \color{} with HTML colorspec
%  \bgcolor
%  \array with options (without options, it's equivalent to the matrix environment)

% Of the standard HTML named colors, white, black, red, green, blue and yellow
% are predefined in the color package. Here are the rest.
\definecolor{aqua}{rgb}{0, 1.0, 1.0}
\definecolor{fuschia}{rgb}{1.0, 0, 1.0}
\definecolor{gray}{rgb}{0.502, 0.502, 0.502}
\definecolor{lime}{rgb}{0, 1.0, 0}
\definecolor{maroon}{rgb}{0.502, 0, 0}
\definecolor{navy}{rgb}{0, 0, 0.502}
\definecolor{olive}{rgb}{0.502, 0.502, 0}
\definecolor{purple}{rgb}{0.502, 0, 0.502}
\definecolor{silver}{rgb}{0.753, 0.753, 0.753}
\definecolor{teal}{rgb}{0, 0.502, 0.502}

% Because of conflicts, \space and \mathop are converted to
% \itexspace and \operatorname during preprocessing.

% itex: \space{ht}{dp}{wd}
%
% Height and baseline depth measurements are in units of tenths of an ex while
% the width is measured in tenths of an em.
\makeatletter
\newdimen\itex@wd%
\newdimen\itex@dp%
\newdimen\itex@thd%
\def\itexspace#1#2#3{\itex@wd=#3em%
\itex@wd=0.1\itex@wd%
\itex@dp=#2ex%
\itex@dp=0.1\itex@dp%
\itex@thd=#1ex%
\itex@thd=0.1\itex@thd%
\advance\itex@thd\the\itex@dp%
\makebox[\the\itex@wd]{\rule[-\the\itex@dp]{0cm}{\the\itex@thd}}}
\makeatother

% \tensor and \multiscript
\makeatletter
\newif\if@sup
\newtoks\@sups
\def\append@sup#1{\edef\act{\noexpand\@sups={\the\@sups #1}}\act}%
\def\reset@sup{\@supfalse\@sups={}}%
\def\mk@scripts#1#2{\if #2/ \if@sup ^{\the\@sups}\fi \else%
  \ifx #1_ \if@sup ^{\the\@sups}\reset@sup \fi {}_{#2}%
  \else \append@sup#2 \@suptrue \fi%
  \expandafter\mk@scripts\fi}
\def\tensor#1#2{\reset@sup#1\mk@scripts#2_/}
\def\multiscripts#1#2#3{\reset@sup{}\mk@scripts#1_/#2%
  \reset@sup\mk@scripts#3_/}
\makeatother

% \slash
\makeatletter
\newbox\slashbox \setbox\slashbox=\hbox{$/$}
\def\itex@pslash#1{\setbox\@tempboxa=\hbox{$#1$}
  \@tempdima=0.5\wd\slashbox \advance\@tempdima 0.5\wd\@tempboxa
  \copy\slashbox \kern-\@tempdima \box\@tempboxa}
\def\slash{\protect\itex@pslash}
\makeatother

% math-mode versions of \rlap, etc
% from Alexander Perlis, "A complement to \smash, \llap, and lap"
%   http://math.arizona.edu/~aprl/publications/mathclap/
\def\clap#1{\hbox to 0pt{\hss#1\hss}}
\def\mathllap{\mathpalette\mathllapinternal}
\def\mathrlap{\mathpalette\mathrlapinternal}
\def\mathclap{\mathpalette\mathclapinternal}
\def\mathllapinternal#1#2{\llap{$\mathsurround=0pt#1{#2}$}}
\def\mathrlapinternal#1#2{\rlap{$\mathsurround=0pt#1{#2}$}}
\def\mathclapinternal#1#2{\clap{$\mathsurround=0pt#1{#2}$}}

% Renames \sqrt as \oldsqrt and redefine root to result in \sqrt[#1]{#2}
\let\oldroot\root
\def\root#1#2{\oldroot #1 \of{#2}}
\renewcommand{\sqrt}[2][]{\oldroot #1 \of{#2}}

% Manually declare the txfonts symbolsC font
\DeclareSymbolFont{symbolsC}{U}{txsyc}{m}{n}
\SetSymbolFont{symbolsC}{bold}{U}{txsyc}{bx}{n}
\DeclareFontSubstitution{U}{txsyc}{m}{n}

% Manually declare the stmaryrd font
\DeclareSymbolFont{stmry}{U}{stmry}{m}{n}
\SetSymbolFont{stmry}{bold}{U}{stmry}{b}{n}

% Manually declare the MnSymbolE font
\DeclareFontFamily{OMX}{MnSymbolE}{}
\DeclareSymbolFont{mnomx}{OMX}{MnSymbolE}{m}{n}
\SetSymbolFont{mnomx}{bold}{OMX}{MnSymbolE}{b}{n}
\DeclareFontShape{OMX}{MnSymbolE}{m}{n}{
    <-6>  MnSymbolE5
   <6-7>  MnSymbolE6
   <7-8>  MnSymbolE7
   <8-9>  MnSymbolE8
   <9-10> MnSymbolE9
  <10-12> MnSymbolE10
  <12->   MnSymbolE12}{}

% Declare specific arrows from txfonts without loading the full package
\makeatletter
\def\re@DeclareMathSymbol#1#2#3#4{%
    \let#1=\undefined
    \DeclareMathSymbol{#1}{#2}{#3}{#4}}
\re@DeclareMathSymbol{\neArrow}{\mathrel}{symbolsC}{116}
\re@DeclareMathSymbol{\neArr}{\mathrel}{symbolsC}{116}
\re@DeclareMathSymbol{\seArrow}{\mathrel}{symbolsC}{117}
\re@DeclareMathSymbol{\seArr}{\mathrel}{symbolsC}{117}
\re@DeclareMathSymbol{\nwArrow}{\mathrel}{symbolsC}{118}
\re@DeclareMathSymbol{\nwArr}{\mathrel}{symbolsC}{118}
\re@DeclareMathSymbol{\swArrow}{\mathrel}{symbolsC}{119}
\re@DeclareMathSymbol{\swArr}{\mathrel}{symbolsC}{119}
\re@DeclareMathSymbol{\nequiv}{\mathrel}{symbolsC}{46}
\re@DeclareMathSymbol{\Perp}{\mathrel}{symbolsC}{121}
\re@DeclareMathSymbol{\Vbar}{\mathrel}{symbolsC}{121}
\re@DeclareMathSymbol{\sslash}{\mathrel}{stmry}{12}
\re@DeclareMathSymbol{\bigsqcap}{\mathop}{stmry}{"64}
\re@DeclareMathSymbol{\biginterleave}{\mathop}{stmry}{"6}
\re@DeclareMathSymbol{\invamp}{\mathrel}{symbolsC}{77}
\re@DeclareMathSymbol{\parr}{\mathrel}{symbolsC}{77}
\makeatother

% \llangle, \rrangle, \lmoustache and \rmoustache from MnSymbolE
\makeatletter
\def\Decl@Mn@Delim#1#2#3#4{%
  \if\relax\noexpand#1%
    \let#1\undefined
  \fi
  \DeclareMathDelimiter{#1}{#2}{#3}{#4}{#3}{#4}}
\def\Decl@Mn@Open#1#2#3{\Decl@Mn@Delim{#1}{\mathopen}{#2}{#3}}
\def\Decl@Mn@Close#1#2#3{\Decl@Mn@Delim{#1}{\mathclose}{#2}{#3}}
\Decl@Mn@Open{\llangle}{mnomx}{'164}
\Decl@Mn@Close{\rrangle}{mnomx}{'171}
\Decl@Mn@Open{\lmoustache}{mnomx}{'245}
\Decl@Mn@Close{\rmoustache}{mnomx}{'244}
\makeatother

% Widecheck
\makeatletter
\DeclareRobustCommand\widecheck[1]{{\mathpalette\@widecheck{#1}}}
\def\@widecheck#1#2{%
    \setbox\z@\hbox{\m@th$#1#2$}%
    \setbox\tw@\hbox{\m@th$#1%
       \widehat{%
          \vrule\@width\z@\@height\ht\z@
          \vrule\@height\z@\@width\wd\z@}$}%
    \dp\tw@-\ht\z@
    \@tempdima\ht\z@ \advance\@tempdima2\ht\tw@ \divide\@tempdima\thr@@
    \setbox\tw@\hbox{%
       \raise\@tempdima\hbox{\scalebox{1}[-1]{\lower\@tempdima\box
\tw@}}}%
    {\ooalign{\box\tw@ \cr \box\z@}}}
\makeatother

% \mathraisebox{voffset}[height][depth]{something}
\makeatletter
\NewDocumentCommand\mathraisebox{moom}{%
\IfNoValueTF{#2}{\def\@temp##1##2{\raisebox{#1}{$\m@th##1##2$}}}{%
\IfNoValueTF{#3}{\def\@temp##1##2{\raisebox{#1}[#2]{$\m@th##1##2$}}%
}{\def\@temp##1##2{\raisebox{#1}[#2][#3]{$\m@th##1##2$}}}}%
\mathpalette\@temp{#4}}
\makeatletter

% udots (taken from yhmath)
\makeatletter
\def\udots{\mathinner{\mkern2mu\raise\p@\hbox{.}
\mkern2mu\raise4\p@\hbox{.}\mkern1mu
\raise7\p@\vbox{\kern7\p@\hbox{.}}\mkern1mu}}
\makeatother

%% Fix array
\newcommand{\itexarray}[1]{\begin{matrix}#1\end{matrix}}
%% \itexnum is a noop
\newcommand{\itexnum}[1]{#1}

%% Renaming existing commands
\newcommand{\underoverset}[3]{\underset{#1}{\overset{#2}{#3}}}
\newcommand{\widevec}{\overrightarrow}
\newcommand{\darr}{\downarrow}
\newcommand{\nearr}{\nearrow}
\newcommand{\nwarr}{\nwarrow}
\newcommand{\searr}{\searrow}
\newcommand{\swarr}{\swarrow}
\newcommand{\curvearrowbotright}{\curvearrowright}
\newcommand{\uparr}{\uparrow}
\newcommand{\downuparrow}{\updownarrow}
\newcommand{\duparr}{\updownarrow}
\newcommand{\updarr}{\updownarrow}
\newcommand{\gt}{>}
\newcommand{\lt}{<}
\newcommand{\map}{\mapsto}
\newcommand{\embedsin}{\hookrightarrow}
\newcommand{\Alpha}{A}
\newcommand{\Beta}{B}
\newcommand{\Zeta}{Z}
\newcommand{\Eta}{H}
\newcommand{\Iota}{I}
\newcommand{\Kappa}{K}
\newcommand{\Mu}{M}
\newcommand{\Nu}{N}
\newcommand{\Rho}{P}
\newcommand{\Tau}{T}
\newcommand{\Upsi}{\Upsilon}
\newcommand{\omicron}{o}
\newcommand{\lang}{\langle}
\newcommand{\rang}{\rangle}
\newcommand{\Union}{\bigcup}
\newcommand{\Intersection}{\bigcap}
\newcommand{\Oplus}{\bigoplus}
\newcommand{\Otimes}{\bigotimes}
\newcommand{\Wedge}{\bigwedge}
\newcommand{\Vee}{\bigvee}
\newcommand{\coproduct}{\coprod}
\newcommand{\product}{\prod}
\newcommand{\closure}{\overline}
\newcommand{\integral}{\int}
\newcommand{\doubleintegral}{\iint}
\newcommand{\tripleintegral}{\iiint}
\newcommand{\quadrupleintegral}{\iiiint}
\newcommand{\conint}{\oint}
\newcommand{\contourintegral}{\oint}
\newcommand{\infinity}{\infty}
\newcommand{\bottom}{\bot}
\newcommand{\minusb}{\boxminus}
\newcommand{\plusb}{\boxplus}
\newcommand{\timesb}{\boxtimes}
\newcommand{\intersection}{\cap}
\newcommand{\union}{\cup}
\newcommand{\Del}{\nabla}
\newcommand{\odash}{\circleddash}
\newcommand{\negspace}{\!}
\newcommand{\widebar}{\overline}
\newcommand{\textsize}{\normalsize}
\renewcommand{\scriptsize}{\scriptstyle}
\newcommand{\scriptscriptsize}{\scriptscriptstyle}
\newcommand{\mathfr}{\mathfrak}
\newcommand{\statusline}[2]{#2}
\newcommand{\tooltip}[2]{#2}
\newcommand{\toggle}[2]{#2}

% Theorem Environments
\theoremstyle{plain}
\newtheorem{theorem}{Theorem}
\newtheorem{lemma}{Lemma}
\newtheorem{prop}{Proposition}
\newtheorem{cor}{Corollary}
\newtheorem*{utheorem}{Theorem}
\newtheorem*{ulemma}{Lemma}
\newtheorem*{uprop}{Proposition}
\newtheorem*{ucor}{Corollary}
\theoremstyle{definition}
\newtheorem{defn}{Definition}
\newtheorem{example}{Example}
\newtheorem*{udefn}{Definition}
\newtheorem*{uexample}{Example}
\theoremstyle{remark}
\newtheorem{remark}{Remark}
\newtheorem{note}{Note}
\newtheorem*{uremark}{Remark}
\newtheorem*{unote}{Note}

%-------------------------------------------------------------------

\begin{document}

%-------------------------------------------------------------------

\section*{spectral sequence}

\hypertarget{context}{}\subsubsection*{{Context}}\label{context}

\hypertarget{homological_algebra}{}\paragraph*{{Homological algebra}}\label{homological_algebra}

[[!include homological algebra - contents]]

\hypertarget{algebraic_topology}{}\paragraph*{{Algebraic topology}}\label{algebraic_topology}

[[!include algebraic topology - contents]]

\hypertarget{stable_homotopy_theory}{}\paragraph*{{Stable Homotopy theory}}\label{stable_homotopy_theory}

[[!include stable homotopy theory - contents]]

\hypertarget{contents}{}\section*{{Contents}}\label{contents}

\noindent\hyperlink{Idea}{Idea}\dotfill \pageref*{Idea} \linebreak
\noindent\hyperlink{definition}{Definition}\dotfill \pageref*{definition} \linebreak
\noindent\hyperlink{spectral_sequence}{Spectral sequence}\dotfill \pageref*{spectral_sequence} \linebreak
\noindent\hyperlink{ConvergenceOfSpectralSequences}{Convergence}\dotfill \pageref*{ConvergenceOfSpectralSequences} \linebreak
\noindent\hyperlink{Boundedness}{Boundedness}\dotfill \pageref*{Boundedness} \linebreak
\noindent\hyperlink{Examples}{Examples}\dotfill \pageref*{Examples} \linebreak
\noindent\hyperlink{SpectralSequenceOfFilteredComplex}{Spectral sequence of a filtered complex}\dotfill \pageref*{SpectralSequenceOfFilteredComplex} \linebreak
\noindent\hyperlink{SpectralSequenceOfADoubleComplex}{Spectral sequence of a double complex}\dotfill \pageref*{SpectralSequenceOfADoubleComplex} \linebreak
\noindent\hyperlink{HyperDerivedFunctors}{Spectral sequences for hyper-derived functors}\dotfill \pageref*{HyperDerivedFunctors} \linebreak
\noindent\hyperlink{GrothendieckSpectralSequence}{Grothendieck spectral sequence}\dotfill \pageref*{GrothendieckSpectralSequence} \linebreak
\noindent\hyperlink{SpecialGrothendieckSpectralSequence}{Special Grothendieck spectral sequences}\dotfill \pageref*{SpecialGrothendieckSpectralSequence} \linebreak
\noindent\hyperlink{LeraySpectralSequence}{Leray spectral sequence}\dotfill \pageref*{LeraySpectralSequence} \linebreak
\noindent\hyperlink{BaseChangeSpectralSequence}{Base change spectral sequence for $Tor$ and $Ext$}\dotfill \pageref*{BaseChangeSpectralSequence} \linebreak
\noindent\hyperlink{HochschildSerreSpectralSequence}{Hochschild-Serre spectral sequence}\dotfill \pageref*{HochschildSerreSpectralSequence} \linebreak
\noindent\hyperlink{exact_couples}{Exact couples}\dotfill \pageref*{exact_couples} \linebreak
\noindent\hyperlink{ListOfExamples}{List of examples}\dotfill \pageref*{ListOfExamples} \linebreak
\noindent\hyperlink{lurie_spectral_sequences}{Lurie spectral sequences}\dotfill \pageref*{lurie_spectral_sequences} \linebreak
\noindent\hyperlink{more}{More}\dotfill \pageref*{more} \linebreak
\noindent\hyperlink{properties}{Properties}\dotfill \pageref*{properties} \linebreak
\noindent\hyperlink{basic_lemmas}{Basic lemmas}\dotfill \pageref*{basic_lemmas} \linebreak
\noindent\hyperlink{FirstQuadrant}{First quadrant spectral sequence}\dotfill \pageref*{FirstQuadrant} \linebreak
\noindent\hyperlink{PropertiesCupProductStructure}{Cup product structure}\dotfill \pageref*{PropertiesCupProductStructure} \linebreak
\noindent\hyperlink{related_concepts}{Related concepts}\dotfill \pageref*{related_concepts} \linebreak
\noindent\hyperlink{references}{References}\dotfill \pageref*{references} \linebreak
\noindent\hyperlink{abelianstable_theory}{Abelian/stable theory}\dotfill \pageref*{abelianstable_theory} \linebreak
\noindent\hyperlink{ReferencesNonabelian}{Nonabelian / unstable theory}\dotfill \pageref*{ReferencesNonabelian} \linebreak
\noindent\hyperlink{history}{History}\dotfill \pageref*{history} \linebreak


\hypertarget{Idea}{}\subsection*{{Idea}}\label{Idea}

The notion of \emph{spectral sequence} is an [[algorithm]] or computational tool in [[homological algebra]] and more generally in [[homotopy theory]] which allows to compute [[chain homology|chain homology groups]]/[[homotopy groups]] of \emph{bi}-[[graded objects]] from the homology/homotopy of the two graded components.

Notably there is a spectral sequence for computing the homology of the [[total complex]] of a [[double complex]] from the homology of its row and column complexes separately. This in turn allows to compute [[derived functors]] of composite functors $G\circ F$ from the double complex $\mathbb{R}^\bullet G (\mathbb{R}^\bullet F(-))$ obtained by non-totally deriving the two functors separately (called the \hyperlink{GrothendieckSpectralSequence}{Grothendieck spectral sequence}). By choosing various functors $F$ and $G$ here this gives rise to various important classes of examples of spectral sequences, see \hyperlink{SpecialGrothendieckSpectralSequence}{below}.

More concretely, a homology spectral sequence is a sequence of graded chain complexes that provides the higher order corrections to the \emph{na\"i{}ve} idea of computing the homology of the [[total complex]] $Tot(V)_\bullet$ of a [[double complex]] $V_{\bullet, \bullet}$: by first computing those of the vertical differential, then those of the horizontal differential induced on these vertical homology groups (or the other way around). This simple idea in general does not produce the correct homology groups of $Tot(V)_\bullet$, but it does produce a ``first-order approximation'' to them, in a useful sense. The spectral sequence is the sequence of higher-order corrections that make this naive idea actually work.

Being, therefore, an iterative perturbative approximation scheme of bigraded differential objects, fully-fledged spectral sequences can look a bit intricate. However, a standard experience in mathematical practice is that for most problems of practical interest the relevant spectral sequence ``perturbation series'' yields the exact result already at the second stage. This reduces the computational complexity immensely and makes spectral sequences a wide-spread useful computational tool.

Despite their name, there seemed to be nothing specifically ``spectral'' about spectral sequences, for any of the technical meanings of the word [[spectrum - disambiguation|spectrum]]. Together with the concept, this term was introduced by [[Jean Leray]] and has long become standard, but was never really motivated (see p. 5 of \hyperlink{Chow}{Chow}). But then, by lucky coincidence it turns out in the refined context of [[stable (∞,1)-category]] theory/[[stable homotopy theory]] that spectral sequences frequently arise by considering the [[homotopy groups]] of \emph{[[sequential diagram|sequences]] of [[spectra]]}. This is discussed at \emph{[[spectral sequence of a filtered stable homotopy type]]}.

While therefore spectral sequences are a notion considered in the context of [[homological algebra]] and more generally in [[stable homotopy theory]], there is also an ``unstable'' or [[nonabelian cohomology|nonabelian]] variant of the notion in plain [[homotopy theory]], called \emph{[[homotopy spectral sequence]]}.

\hypertarget{definition}{}\subsection*{{Definition}}\label{definition}

We give the general definition of a (co)homology spectral sequence. For motivation see the example \emph{\hyperlink{SpectralSequenceOfFilteredComplex}{Spectral sequence of a filtered complex}} below.

Throughout, let $\mathcal{A}$ be an [[abelian category]].

\hypertarget{spectral_sequence}{}\subsubsection*{{Spectral sequence}}\label{spectral_sequence}

\begin{defn}
\label{cohomology_spectral_sequence}\hypertarget{}{}
A \textbf{cohomology spectral sequence} in $\mathcal{A}$ is

\begin{itemize}%
\item a family $(E^{p,q}_r)$ of [[objects]] in $\mathcal{A}$, for all [[integers]] $p,q,r$ with $r\geq 1$

(for a fixed $r$ these are said to form the \textbf{$r$-th page} of the spectral sequence)


\item for each $p,q,r$ as above a [[morphism]] (called the \textbf{differential})

\begin{displaymath}
d^{p,q}_r:E^{p,q}_r\to E^{p+r,q-r+1}_r
\end{displaymath}
satisfying $d_r^2 = 0$ (more precisely, $d_r^{p+r,q-r+1}\circ d_r^{p,q} = 0$)


\item [[isomorphisms]] $\alpha_r^{p,q}: H^{p,q}(E_r)\to E^{p,q}_{r+1}$ where the [[chain cohomology]] is given by

\begin{displaymath}
H^{p,q}(E_r) = \mathrm{ker} d^{p,q}_r/ \mathrm{im} d^{p-r,q+r-1}_r
 \,.
\end{displaymath}


\end{itemize}
\end{defn}
Analogously a \textbf{homology spectral sequence} is collection of objects $(E_{p,q}^r)$ with the differential $d_r$ of degree $(-r,r-1)$.

\hypertarget{ConvergenceOfSpectralSequences}{}\subsubsection*{{Convergence}}\label{ConvergenceOfSpectralSequences}

\begin{defn}
\label{LimitTerm}\hypertarget{LimitTerm}{}
Let $\{E^r_{p,q}\}_{r,p,q}$ be a [[spectral sequence]] such that for each $p,q$ there is $r(p,q)$ such that for all $r \geq r(p,q)$ we have

\begin{displaymath}
E^{r \geq r(p,q)}_{p,q} \simeq E^{r(p,q)}_{p,q}
  \,.
\end{displaymath}
Then one says equivalently that

\begin{enumerate}%
\item the [[bigraded object]]

\begin{displaymath}
E^\infty 
   \coloneqq 
  \{E^\infty_{p,q}\}_{p,q} \coloneqq \{ E^{r(p,q)}_{p,q} \}_{p,q}
\end{displaymath}
is the \textbf{limit term} of the spectral sequence;


\item the spectral sequence \textbf{abuts} to $E^\infty$.



\end{enumerate}
\end{defn}
\begin{example}
\label{Degeneration}\hypertarget{Degeneration}{}
If for a spectral sequence there is $r_s$ such that all [[differentials]] on pages after $r_s$ vanish, $\partial^{r \geq r_s} = 0$, then $\{E^{r_s}\}_{p,q}$ is limit term for the spectral sequence. One says in this cases that the spectral sequence \textbf{degenerates} at $r_s$.

\end{example}
\begin{proof}
By the defining relation

\begin{displaymath}
E^{r+1}_{p,q} \simeq ker(\partial^r_{p-r,q+r-1})/im(\partial^r_{p,q}) = E^r_{pq}
\end{displaymath}
the spectral sequence becomes constant in $r$ from $r_s$ on if all the differentials vanish, so that $ker(\partial^r_{p,q}) = E^r_{p,q}$ for all $p,q$.

\end{proof}
\begin{example}
\label{Collaps}\hypertarget{Collaps}{}
If for a [[spectral sequence]] $\{E^r_{p,q}\}_{r,p,q}$ there is $r_s \geq 2$ such that the $r_s$th page is concentrated in a single row or a single column, then the the spectral sequence degenerates on this pages, example \ref{Degeneration}, hence this page is a limit term, def. \ref{LimitTerm}. One says in this case that the spectral sequence \textbf{collapses} on this page.

\end{example}
\begin{proof}
For $r \geq 2$ the [[differentials]] of the spectral sequence

\begin{displaymath}
\partial^r \colon E^r_{p,q} \to E^r_{p-r, q+r-1}
\end{displaymath}
have [[domain]] and [[codomain]] necessarily in different rows an columns (while for $r = 1$ both are in the same row and for $r = 0$ both coincide). Therefore if all but one row or column vanish, then all these differentials vanish.

\end{proof}
\begin{defn}
\label{Convergence}\hypertarget{Convergence}{}
A [[spectral sequence]] $\{E^r_{p,q}\}_{r,p,q}$ is said to \textbf{converge} to a [[graded object]] $H_\bullet$ with [[filtered chain complex|filtering]] $F_\bullet H_\bullet$, traditionally denoted

\begin{displaymath}
E^r_{p,q} \Rightarrow H_\bullet
  \,,
\end{displaymath}
if the [[associated graded]] complex $\{G_p H_{p+q}\}_{p,q} \coloneqq \{F_p H_{p+q} / F_{p-1} H_{p+q}\}$ of $H$ is the limit term of $E$, def. \ref{LimitTerm}:

\begin{displaymath}
E^\infty_{p,q} \simeq G_p H_{p+q} \;\;\;\;\;\;\; \forall_{p,q}
  \,.
\end{displaymath}
\end{defn}
(See also \emph{[[conditional convergence]]}.)

\begin{remark}
%\label{}\hypertarget{}{}
In practice spectral sequences are often referred to via their first non-trivial page, often also the page at which it collapses, def. \ref{Collaps}, often the second page. Then one often uses notation such as

\begin{displaymath}
E^2_{p,q} \Rightarrow H_\bullet
\end{displaymath}
to be read as ``There is a spectral sequence whose second page is as shown on the left and which converges to a filtered object as shown on the right.''

\end{remark}
\begin{remark}
%\label{}\hypertarget{}{}
In applications one is interested in computing the $H_n$ and uses spectral sequences converging to this as tools for approximating $H_n$ in terms of the given filtration.

Therefore usually spectral sequences are required to converge in each degree, or even that for each pair $(p,q)$ there exists an $r_0$ such that for all $r\geq r_0$, $d_r^{p-r,q+r-1} = 0$.

\end{remark}
\begin{remark}
%\label{}\hypertarget{}{}
If $(E^r)$ collapses at $r$, then it converges to $H_\bullet$ with $H_n$ being the unique entry $E_{p,q}^r$ on the non-vanishing row/column with $p+q = n$.

\end{remark}
\hypertarget{Boundedness}{}\subsubsection*{{Boundedness}}\label{Boundedness}

\begin{defn}
\label{BoundedSpectralSequence}\hypertarget{BoundedSpectralSequence}{}
A spectral sequence $\{E^r_{p,q}\}$ is called a \textbf{bounded spectral sequence} if for all $n,r \in \mathbb{Z}$ the number of non-vanishing terms of the form $E^r_{k,n-k}$ is finite.

\end{defn}
\begin{defn}
\label{QuadrantSpectralSequence}\hypertarget{QuadrantSpectralSequence}{}
A [[spectral sequence]] $\{E^r_{p,q}\}$ is called

\begin{itemize}%
\item a \textbf{first quadrant spectral sequence} if all terms except possibly for $p,q \geq 0$ vanish;


\item a \textbf{third quadrant spectral sequence} if all terms except possibly for $p,q \leq 0$ vanish.



\end{itemize}
Such spectral sequences are bounded, def. \ref{BoundedSpectralSequence}.

\end{defn}
\begin{prop}
\label{BoundedSpectralSequenceHasLimitTerm}\hypertarget{BoundedSpectralSequenceHasLimitTerm}{}
A bounded spectral sequence, def. \ref{BoundedSpectralSequence}, has a limit term, def. \ref{LimitTerm}.

\end{prop}
\begin{proof}
First notice that if a spectral sequence has at most $N$ non-vanishing terms of total degree $n$ on page $r$, then all the following pages have at most at these positions non-vanishing terms, too, since these are the homologies of the previous terms.

Therefore for a bounded spectral sequence for each $n$ there is $L(n) \in \mathbb{Z}$ such that $E^r_{p,n-p} = 0$ for all $p \leq L(n)$ and all $r$. Similarly there is $T(n) \in \mathbb{Z}$ such $E^r_{n-q,q} = 0$ for all $q \leq T(n)$ and all $r$.

We claim then that the limit term of the bounded spectral sequence is in position $(p,q)$ given by the value $E^r_{p,q}$ for

\begin{displaymath}
r \gt max(  p-L(p+q-1), q + 1 - L(p+q+1) )
  \,.
\end{displaymath}
This is because for such $r$ we have

\begin{enumerate}%
\item $E^r_{p-r, q+r-1} = 0$ because $p-r \lt L(p+q-1)$, and hence the [[kernel]] $ker(\partial^r_{p-r,q+r-1}) = 0$ vanishes;


\item $E^r_{p+r, q-r+1} = 0$ because $q-r + 1 \lt T(p+q+1)$, and hence the [[image]] $im(\partial^r_{p,q}) = 0$ vanishes.



\end{enumerate}
Therefore

\begin{displaymath}
\begin{aligned}
    E^{r+1}_{p,q} 
    &= 
    ker(\partial^r_{p-r,q+r-1})/im(\partial^r_{p,q})
    \\
    & \simeq E^r_{p,q}/0
    \\
    & \simeq E^r_{p,q}
  \end{aligned}
  \,.
\end{displaymath}
\end{proof}
\hypertarget{Examples}{}\subsection*{{Examples}}\label{Examples}

The basic class of examples are

\begin{itemize}%
\item \hyperlink{SpectralSequenceOfFilteredComplex}{Spectral sequences of filtered complexes}

\end{itemize}
which compute the cohomology of a [[filtered object|filtered complex]] from the cohomologies of its [[associated graded objects]].

From this one obtains as a special case the class of

\begin{itemize}%
\item \hyperlink{SpectralSequenceOfADoubleComplex}{Spectral sequences of double complexes}

\end{itemize}
which compute the cohomology of the [[total complex]] of a [[double complex]] using the two canonical filtrations of this by row- and by column-degree.

From this in turn one obtains as a special case the class of

\begin{itemize}%
\item \hyperlink{GrothendieckSpectralSequence}{Grothendieck spectral sequences}

\end{itemize}
which compute the [[derived functor]] $\mathbb{R}^\bullet(G \circ F (-))$ of the composite of two functors from the spectral sequence of the double complex $\mathbb{R}^\bullet (F (\mathbb{R}^\bullet G (-)))$.

Many special cases of this for various choices of $F$ and $G$ go by special names, this we tabulate at

\begin{itemize}%
\item \emph{\hyperlink{ListOfExamples}{List of examples}}.

\end{itemize}
\hypertarget{SpectralSequenceOfFilteredComplex}{}\subsubsection*{{Spectral sequence of a filtered complex}}\label{SpectralSequenceOfFilteredComplex}

The fundamental example of a spectral sequence, from which essentially all the other examples arise as special cases, is the \emph{[[spectral sequence of a filtered complex]]}. (See there for details). Or more generally in [[stable homotopy theory]]: the [[spectral sequence of a filtered stable homotopy type]].

If a [[cochain complex]] $C^\bullet$ is equipped with a [[filtered object|filtration]] $F^\bullet C^\bullet$, there is an induced filtration $F^\bullet H(C)$ of its [[cohomology groups]], according to which levels of the filtration contain representatives for the various cohomology classes.

A filtration $F$ also gives rise to an [[associated graded object]] $Gr(F)$, whose grades are the successive level inclusion [[cokernels]]. Generically, the operations of grading and cohomology do not commute:

\begin{displaymath}
Gr(F^\bullet H^\bullet(C)) \neq H^\bullet (Gr(F^\bullet) C)
  \,.
\end{displaymath}
But the [[spectral sequence of a filtered complex|spectral sequence associated to a filtered complex]] $F^\bullet C^\bullet$, passes through $H^\bullet (Gr(F^\star) C)$ in the page $E_{(1)}$ and in good cases converges to $Gr(F^* H^\bullet(C))$.

\hypertarget{SpectralSequenceOfADoubleComplex}{}\subsubsection*{{Spectral sequence of a double complex}}\label{SpectralSequenceOfADoubleComplex}

The [[total complex]] of a [[double complex]] is naturally filtered in two ways: by columns and by rows. By the above [[spectral sequence of a filtered complex]] this gives two different spectral sequences associated computing the cohomology of a double complex from the cohomologies of its rows and columns. Many other classes of spectral sequences are special cases of this cases, notably the [[Grothendieck spectral sequence]] and \emph{its} special cases.

This is discussed at \emph{[[spectral sequence of a double complex]]}.

\hypertarget{HyperDerivedFunctors}{}\subsubsection*{{Spectral sequences for hyper-derived functors}}\label{HyperDerivedFunctors}

From the [[spectral sequence for a double complex]] one obtains as a special case a spectral sequence that computes [[hyper-derived functors]].

(\ldots{})

\hypertarget{GrothendieckSpectralSequence}{}\subsubsection*{{Grothendieck spectral sequence}}\label{GrothendieckSpectralSequence}

The \emph{[[Grothendieck spectral sequence]]} computes the composite of two [[derived functors]] from the two derived functors separately.

Let $\mathcal{A} \stackrel{F}{\to} \mathcal{B} \stackrel{G}{\to} \mathcal{C}$ be two [[left exact functors]] between [[abelian categories]].

Write $R^p F : \mathcal{D} \to Ab$ for the [[cochain cohomology]] of the [[derived functor]] of $F$ in degree $p$ etc. .

\begin{theorem}
\label{GrothendieckSpectralSequence}\hypertarget{}{}
If $F$ sends [[injective objects]] of $\mathcal{A}$ to $G$-[[acyclic objects]] in $\mathcal{B}$ then for each $A \in \mathcal{A}$ there is a \hyperlink{FirstQuadrant}{first quadrant} cohomology spectral sequence

\begin{displaymath}
E_r^{p,q} := (R^p G \circ R^q F)(A)
\end{displaymath}
that converges to the right [[derived functor]] of the composite functor

\begin{displaymath}
E_r^{p,q} \Rightarrow R^{p+q} (G \circ F)(A).
\end{displaymath}
Moreover

\begin{enumerate}%
\item the [[edge maps]] in this spectral sequence are the canonical morphisms

\begin{displaymath}
R^p G (F A) \to R^p (G \circ F)(A)
\end{displaymath}
induced from applying $F$ to an injective resolution $A \to \hat A$ and the morphism

\begin{displaymath}
R^q (G \circ F)(A) \to G(R^q F (A))
  \,.
\end{displaymath}

\item the [[exact sequence]] of low degree terms is

\begin{displaymath}
0 \to (R^1 G)(F(A)) \to R^1(G \circ F)(A)
   \to G(R^1(F(A)))
   \to (R^2 G)(F(A))
   \to R^2(G \circ F)(A)
\end{displaymath}


\end{enumerate}
\end{theorem}
This is called the \emph{[[Grothendieck spectral sequence]]}.

\begin{proof}
Since for $A \to \hat A$ an injective [[resolution]] of $A$ the complex $F(\hat A)$ is a chain complex not concentrated in a single degree, we have that $R^p (G \circ F)(A)$ is equivalently the [[hyper-derived functor]] evaluation $\mathbb{R}^p(G) (F(A))$.

Therefore the second spectral sequence discussed at \hyperlink{HyperDerivedFunctors}{hyper-derived functor spectral sequences} converges as

\begin{displaymath}
(R^p G)H^q(F(\hat A)) \Rightarrow R^p (G \circ F)(A)
  \,.
\end{displaymath}
Now since by construction $H^q(F(\hat A)) = R^q F(A)$ this is a spectral sequence

\begin{displaymath}
(R^p G)(R^q F) A) \Rightarrow R^p (G \circ F)(A)
  \,.
\end{displaymath}
This is the Grothendieck spectral sequence.

\end{proof}
\hypertarget{SpecialGrothendieckSpectralSequence}{}\subsubsection*{{Special Grothendieck spectral sequences}}\label{SpecialGrothendieckSpectralSequence}

\begin{itemize}%
\item \hyperlink{LeraySpectralSequence}{Leray spectral sequence}


\item \hyperlink{HochschildSerreSpectralSequence}{Hochschild-Serre spectral sequence}


\item \hyperlink{BaseChangeSpectralSequence}{Base-change spectral sequence}



\end{itemize}
\hypertarget{LeraySpectralSequence}{}\paragraph*{{Leray spectral sequence}}\label{LeraySpectralSequence}

The \emph{[[Leray spectral sequence]]} is the special case of the [[Grothendieck spectral sequence]] for the case where the two functors being composed are a [[direct image|push-forward]] of [[sheaves]] of [[abelian groups]] along a [[continuous map]] $f : X \to Y$ followed by the push-forward $X \to *$ to the point. This yields a spectral sequence that computes the [[abelian sheaf cohomology]] on $X$ in terms of the abelian sheaf cohomology on $Y$.

\begin{theorem}
%\label{}\hypertarget{}{}
Let $X, Y$ be suitable [[sites]] and $f : X \to Y$ be a morphism of sites. () Let $\mathcal{C} = Ch_\bullet(Sh(X,Ab))$ and $\mathcal{D} = Ch_\bullet(Sh(Y,Ab))$ be the [[model structure on chain complexes|model categories of complexes of sheaves of abelian groups]]. The [[direct image]] $f_*$ and [[global section]] functor $\Gamma_Y$ compose to $\Gamma_X$:

\begin{displaymath}
\Gamma_X : \mathcal{C} \stackrel{f_*}{\to} \mathcal{D} \stackrel{\Gamma_Y}{\to} Ch_\bullet(Ab)
  \,.
\end{displaymath}
Then for $A \in Sh(X,Ab)$ a sheaf of abelian groups on $X$ there is a cohomology spectral sequence

\begin{displaymath}
E_r^{p,q} := H^p(Y, R^q f_* A)
\end{displaymath}
that converges as

\begin{displaymath}
E_r^{p,q} \Rightarrow H^{p+q}(X, A)
\end{displaymath}
and hence computes the cohomology of $X$ with coefficients in $A$ in terms of the cohomology of $Y$ with coefficients in the push-forward of $A$.

\end{theorem}
\hypertarget{BaseChangeSpectralSequence}{}\paragraph*{{Base change spectral sequence for $Tor$ and $Ext$}}\label{BaseChangeSpectralSequence}

For $R$ a [[ring]] write $R$[[Mod]] for its category of [[modules]]. Given a [[homomorphism]] of [[ring]] $f : R_1 \to R_2$ and an $R_2$-[[module]] $N$ there are composites of [[base change]] along $f$ with the [[hom-functor]] and the [[tensor product]] functor

\begin{displaymath}
R_1 Mod \stackrel{\otimes_{R_1} R_2}{\to} R_2 Mod \stackrel{\otimes_{R_2} N}{\to} Ab
\end{displaymath}
\begin{displaymath}
R_1 Mod \stackrel{Hom_{R_1 Mod}(-,R_2)}{\to}
  R_2 Mod
  \stackrel{Hom_{R_2}(-,N)}{\to}
  Ab
  \,.
\end{displaymath}
The [[derived functors]] of $Hom_{R_2}(-,N)$ and $\otimes_{R_2} N$ are the [[Ext]]- and the [[Tor]]-functors, respectively, so the [[Grothendieck spectral sequence]] applied to these composites yields [[base change spectral sequence]] for these.

\hypertarget{HochschildSerreSpectralSequence}{}\paragraph*{{Hochschild-Serre spectral sequence}}\label{HochschildSerreSpectralSequence}

\begin{itemize}%
\item [[Hochschild-Serre spectral sequence]]

\end{itemize}
\hypertarget{exact_couples}{}\subsubsection*{{Exact couples}}\label{exact_couples}

The above examples are all built on the \hyperlink{SpectralSequenceOfFilteredComplex}{spectral sequence of a filtered complex}. An alternatively universal construction builds spectral sequences from \emph{exact couples}.

An \textbf{[[exact couple]]} is an exact sequence of three arrows among two objects

\begin{displaymath}
E \overset{j}{\to} D \overset{\varphi}{\to} D \overset{k}{\to} E \overset{j}{\to}.
\end{displaymath}
These creatures construct spectral sequences by a two-step process:

\begin{itemize}%
\item first, the composite $d \coloneqq k j \colon E\to E$ is nilpotent, in that $d^2=0$
\item second, the homology $E'$ of $(E,d)$ supports a map $j':E'\to \varphi D$, and receives a map $k':\varphi D\to E'$. Setting $D'=\varphi D$, by general reasoning

\end{itemize}
\begin{displaymath}
E' \overset{j'}{\to} D' \overset{\varphi}{\to} D' \overset{k'}{\to} E' \overset{j'}{\to}
  \,.
\end{displaymath}
is again an exact couple.

The sequence of complexes $(E,d),(E',d'),\dots$ is a spectral sequence, by construction.

Examples of exact couples can be constructed in a number of ways. Importantly, any short exact sequence involving two distinct chain complexes provides an exact couple among their total homology complexes, via the Mayer-Vietoris long exact sequence; in particular, applying this procedure to the relative homology of a filtered complex gives precisely the spectral sequence of the filtered complex described (???) somewhere else on this page. For another example, choosing a chain complex of flat modules $(C^\bullet,d)$, tensoring with the short exact sequence

\begin{displaymath}
\mathbb{Z}/p\mathbb{Z} \to \mathbb{Z}/p^2\mathbb{Z} \to \mathbb{Z}/p\mathbb{Z}
\end{displaymath}
gives the exact couple

\begin{displaymath}
H^\bullet(d,\mathbb{Z}/p^2\mathbb{Z})
    \overset{[\cdot]}{\to} 
  H^\bullet(d,\mathbb{Z}/p\mathbb{Z})
   \overset{\beta}{\to}
  H^\bullet(d,\mathbb{Z}/p\mathbb{Z})
   \overset{p}{\to}H^\bullet(d,\mathbb{Z}/p^2\mathbb{Z})\cdots
\end{displaymath}
in which $\beta$ is the \emph{mod-$p$ Bockstein} homomorphism.

The exact couple recipe for spectral sequences is notable in that it doesn't mention any grading on the objects $D,E$; trivially, an exact couple can be specified by a short exact sequence $\text{coker} \varphi\to E\to \ker\varphi$, although this obscures the focus usually given to $E$. In applications, a bi-grading is usually induced by the context, which also specifies bidegrees for the initial maps $j,k,\varphi$, leading to the conventions mentioned earlier.

\hypertarget{ListOfExamples}{}\subsubsection*{{List of examples}}\label{ListOfExamples}

\hypertarget{lurie_spectral_sequences}{}\paragraph*{{Lurie spectral sequences}}\label{lurie_spectral_sequences}

[[!include Lurie spectral sequences -- table]]

\hypertarget{more}{}\paragraph*{{More}}\label{more}

The following list of examples orders the various classes of spectral sequences by special cases: items further to the right are special cases of items further to the left.

\begin{itemize}%
\item \textbf{\emph{[[spectral sequence of a filtered complex]]}} (approximate [[chain homology]] by higher-order [[relative homology]] in the presence of a [[filtered object|filtering]])

\begin{itemize}%
\item [[Atiyah-Hirzebruch spectral sequence]] (compute [[generalized homology]] of a [[Serre fibration]] in terms of that of the base with [[coefficients]] in that of the [[fibers]])

\begin{itemize}%
\item [[Serre spectral sequence]] (compute [[ordinary homology]] of a [[Serre fibration]] in terms of that of the base with [[coefficients]] in that of the [[fibers]])

\end{itemize}

\item [[Eilenberg-Moore spectral sequence]] (compute homology of [[homotopy fiber products]])


\item \textbf{[[spectral sequence of a double complex]]} (compute homology of [[total complex]] by filtering by row/column degree)

\begin{itemize}%
\item \textbf{[[Grothendieck spectral sequence]]} (compute [[composition]] of two [[derived functors in homological algebra|derived functors]])

\begin{itemize}%
\item [[base change spectral sequence]] (compute [[base change]]/[[extension of scalars]] in two stages)


\item [[Leray spectral sequence]] (compute [[global sections]] in two stages)


\item [[Hochschild-Serre spectral sequence]] (compute [[group cohomology]] in two stages)



\end{itemize}


\end{itemize}


\end{itemize}


\end{itemize}
Here is a more random list (using material from \hyperlink{Wikipedia}{Wikipedia}). Eventually to be merged with the above.

\begin{itemize}%
\item [[Adams spectral sequence]] in [[stable homotopy theory]]


\item [[Adams–Novikov spectral sequence]], converging to [[homotopy groups]] of [[connective spectra]]


\item [[chromatic spectral sequence]] for calculating the initial terms of the [[Adams–Novikov spectral sequence]]


\item [[Atiyah–Hirzebruch spectral sequence]] of an [[extraordinary cohomology theory]];


\item [[descent spectral sequence]]


\item [[Bar spectral sequence]] for the [[homology]] of the [[classifying space]] of a [[group]]


\item [[Barratt spectral sequence]] converging to the [[homotopy]] of the initial space of a [[cofibration]]


\item [[Bloch–Lichtenbaum spectral sequence]] converging to the [[algebraic K-theory]] of a [[field]]


\item [[Bockstein spectral sequence]] relating the [[homology]] with $mod p$ coefficients and the homology reduced $mod p$


\item [[Bousfield–Kan spectral sequence]] converging to the [[homotopy colimit]] of a [[functor]]


\item [[Cartan–Leray spectral sequence]] converging to the [[homology]] of a [[quotient space]]


\item [[Cech-to-derived functor spectral sequence]] from [[Čech cohomology]] to [[abelian sheaf cohomology]]


\item [[change of rings spectral sequences]] for calculating [[Tor]] and [[Ext]] groups of modules


\item [[Connes spectral sequences]] converging to the [[cyclic homology]] of an algebra


\item [[EHP spectral sequence]] converging to [[stable homotopy groups of spheres]]


\item [[Eilenberg–Moore spectral sequence]] for the [[singular cohomology]] of the [[pullback]] of a [[fibration]]


\item [[Federer spectral sequence]] converging to [[homotopy groups]] of a function space


\item [[Frölicher spectral sequence]] starting from the [[Dolbeault cohomology]] and converging to the algebraic [[de Rham cohomology]] of a [[variety]]


\item [[Green's spectral sequence]] for [[Koszul cohomology]]


\item [[Grothendieck spectral sequence]] for composing [[derived functors]]


\item [[Hodge–de Rham spectral sequence]] converging to the algebraic [[de Rham cohomology]] of a [[variety]]


\item [[Hurewicz spectral sequence]] for calculating the [[homology]] of a space from its [[homotopy groups]]


\item [[hyperhomology spectral sequence]] for calculating [[hyperhomology]]


\item [[Künneth spectral sequence]] for calculating the homology of a [[tensor product]] of [[differential graded algebras]]


\item [[Leray spectral sequence]] converging to [[abelian sheaf cohomology]]


\item [[Leray–Serre spectral sequence]] of a [[fibration]]


\item [[Lyndon–Hochschild–Serre spectral sequence]] in [[group cohomology]]


\item [[May spectral sequence]] for calculating the [[Tor]] or [[Ext]] groups of an algebra


\item [[Miller spectral sequence]] converging to the $mod p$ [[stable homology]] of a space


\item [[Quillen spectral sequence]] for calculating the homotopy of a [[simplicial group]]


\item [[spectral sequence of an exact couple]]


\item [[van Est spectral sequence]] converging to relative [[Lie algebra cohomology]]


\item [[van Kampen spectral sequence]] for calculating the homotopy of a wedge of spaces.


\item [[slice spectral sequence]]



\end{itemize}
\hypertarget{properties}{}\subsection*{{Properties}}\label{properties}

\hypertarget{basic_lemmas}{}\subsubsection*{{Basic lemmas}}\label{basic_lemmas}

\begin{lemma}
\label{mapping_lemma}\hypertarget{}{}
\textbf{(mapping lemma)}

If $f : (E_r^{p,q} \to (F_r^{p,q}))$ is a morphism of spectral sequences such that for some $r$ we have that $f_r : E_r^{p,q} \to F_r^{p,q}$ is an isomorphism, then also $f_s$ is an isomorphism for all $s \geq r$.

\end{lemma}
\begin{lemma}
\label{classical_convergence_theorem}\hypertarget{}{}
\textbf{(classical convergence theorem)}

(\ldots{})

\end{lemma}
This is recalled in (\hyperlink{Weibel}{Weibel, theorem 5.51}).

\hypertarget{FirstQuadrant}{}\subsubsection*{{First quadrant spectral sequence}}\label{FirstQuadrant}

\begin{defn}
\label{first_quadrant_spectral_sequence}\hypertarget{}{}
A \textbf{first quadrant spectral sequence} is one for wich all pages are concentrated in the first quadrant of the $(p,q)$-plane, in that

\begin{displaymath}
((p \lt 0) or (q \lt 0))
  \;\;
  \Rightarrow
  E_r^{p,q} = 0
  \,.
\end{displaymath}
\end{defn}
\begin{prop}
%\label{}\hypertarget{}{}
If the $r$th page is concentrated in the first quadrant, then so the $(r+1)st$ page. So if the first one is, then all are.

\end{prop}
\begin{prop}
%\label{}\hypertarget{}{}
Every first quadrant spectral sequence converges at $(p,q)$ from $r \gt max(p,q+1)$ on

\begin{displaymath}
E_{max(p,q+1)+1}^{p,q} = E_\infty^{p,q}
  \,.
\end{displaymath}
\end{prop}
\begin{prop}
%\label{}\hypertarget{}{}
If a first quadrant spectral sequence converges

\begin{displaymath}
E_r^{p,q} \Rightarrow H^{p+q}
\end{displaymath}
then each $H^n$ has a filtration of length $n+1$

\begin{displaymath}
0 = F^{n+1}H^n \subset F^n H^n \subset \cdots \subset F^1 H^n \subset F^0 H^n = H^n
\end{displaymath}
and we have

\begin{itemize}%
\item $F^n H^n \simeq E_\infty^{n,0}$


\item $H^n/F^1 H^n \simeq E_\infty^{0,n}$.



\end{itemize}
\end{prop}
\hypertarget{PropertiesCupProductStructure}{}\subsubsection*{{Cup product structure}}\label{PropertiesCupProductStructure}

Cohomological spectral sequences are compatible with [[cup product]] structure on the $E_2$-page. (e.g. \hyperlink{Hutchings11}{Hutchings 11, sections 5 and 6})

\hypertarget{related_concepts}{}\subsection*{{Related concepts}}\label{related_concepts}

\begin{itemize}%
\item [[edge morphism]]

\end{itemize}
\hypertarget{references}{}\subsection*{{References}}\label{references}

\begin{itemize}%
\item [[Frank Adams]], part III section 7 of \emph{[[Stable homotopy and generalised homology]]}, 1974

\end{itemize}
\hypertarget{abelianstable_theory}{}\subsubsection*{{Abelian/stable theory}}\label{abelianstable_theory}

An elementary pedagogical introduction is in

\begin{itemize}%
\item [[Timothy Chow]], \emph{You could have invented spectral sequences}, Notices of the AMS (2006) (\href{http://www.ams.org/notices/200601/fea-chow.pdf}{pdf})

\end{itemize}
Standard textbook references are

\begin{itemize}%
\item [[John McCleary]], \emph{A User's Guide to Spectral Sequences}, Cambridge University Press

\end{itemize}
chapter 5 of

\begin{itemize}%
\item [[Charles Weibel]], \emph{An introduction to homological algebra} Cambridge studies in advanced mathematics 38 (1994)

\end{itemize}
and section 14 of

\begin{itemize}%
\item [[Raoul Bott]], [[Loring Tu]], \emph{Differential forms in algebraic topology}, Graduate Texts in Mathematics \textbf{82}, Springer 1982. xiv+331 pp.

\end{itemize}
and

\begin{itemize}%
\item [[Hal Schenck]], \emph{Chapter 9: Cohomology and spectral sequences} (\href{http://www.math.uiuc.edu/~schenck/tapp.pdf}{pdf}) .

\end{itemize}
A textbook with a focus on applications in [[algebraic topology]] is

\begin{itemize}%
\item [[Alan Hatcher]], \emph{Spectral sequences in algebraic topology} (\href{http://www.math.cornell.edu/~hatcher/SSAT/SSATpage.html}{web})

\end{itemize}
The general discussion in the context of [[stable (∞,1)-category]] theory (the [[spectral sequence of a filtered stable homotopy type]]) is in section 1.2.2 of

\begin{itemize}%
\item [[Jacob Lurie]], \emph{[[Higher Algebra]]}

\end{itemize}
A review Master thesis is

\begin{itemize}%
\item Jennifer Orlich, \emph{Spectral sequences and an application} (\href{http://www.math.osu.edu/~flicker.1/orlich.pdf}{pdf})

\end{itemize}
Reviews of and lecture notes on standard definitions and facts about spectral sequences include

\begin{itemize}%
\item Matthew Greenberg, \emph{Spectral sequences} (\href{http://www.math.mcgill.ca/goren/SeminarOnCohomology/specseq.pdf}{pdf})


\item [[Michael Hutchings]], \emph{Introduction to spectral sequences} (\href{http://math.berkeley.edu/~hutching/teach/215b-2011/ss.pdf}{pdf})


\item Daniel Murfet, \emph{Spectral sequences} (\href{http://therisingsea.org/notes/SpectralSequences.pdf}{pdf})


\item [[Neil Strickland]], \emph{Spectral sequences} (\href{http://neil-strickland.staff.shef.ac.uk/courses/bestiary/ss.pdf}{pdf})


\item Ravi Vakil, \emph{Spectral Sequences: Friend or Foe?} (\href{http://math.stanford.edu/~vakil/0708-216/216ss.pdf}{pdf})


\item Brandon Williams, \emph{Spectral sequences} (\href{http://www.math.sunysb.edu/~mbw/notes/orals/Spectral%20Sequences.pdf}{pdf})



\end{itemize}
Original articles incluce

\begin{itemize}%
\item [[Michael Boardman]], \emph{Conditionally convergent spectral sequences} (\href{http://hopf.math.purdue.edu/Boardman/ccspseq.pdf}{pdf})

\end{itemize}
See also

\begin{itemize}%
\item Wikipedia, \emph{\href{http://en.wikipedia.org/wiki/Spectral_sequence}{Spectral sequence}}

\end{itemize}
\begin{itemize}%
\item A. Romero, J. Rubio, F. Sergeraert, \emph{Computing spectral sequences} (\href{http://www-fourier.ujf-grenoble.fr/~sergerar/Papers/Ana-JSC.pdf}{pdf})


\item Eric Peterson, \emph{\href{http://ext-chart.org}{Ext chart}} software for computing spectral sequences



\end{itemize}
\hypertarget{ReferencesNonabelian}{}\subsubsection*{{Nonabelian / unstable theory}}\label{ReferencesNonabelian}

Homotopy spectral sequences in model categories are discussed in

\begin{itemize}%
\item [[Aldridge Bousfield]], \emph{Cosimplicial resolutions and homotopy spectral sequences in model categories} (\href{http://arxiv.org/abs/math/0312531}{arXiv:math/0312531}).

\end{itemize}
Spectral sequences in general categories with [[zero morphisms]] are discussed in

\begin{itemize}%
\item [[Marco Grandis]], \emph{Homotopy spectral sequences} (\href{http://arxiv.org/abs/1007.0632}{arXiv:1007.0632})

\end{itemize}
Discussion in [[homotopy type theory]] is in

\begin{itemize}%
\item [[Mike Shulman]], \emph{Spectral sequences} 2013 (\href{https://golem.ph.utexas.edu/category/2013/08/what_is_a_spectral_sequence.html}{part I}, \href{http://homotopytypetheory.org/2013/08/08/spectral-sequences/}{part II})

\end{itemize}
\hypertarget{history}{}\subsubsection*{{History}}\label{history}

\begin{itemize}%
\item [[John McCleary]], \emph{A history of spectral sequences: Origins to 1953}, in \emph{History of Topology}, edited by Ioan M. James, North Holland (1999) 631–663

\end{itemize}
[[!redirects spectral sequences]]



\end{document}