\documentclass[11pt,draft,makeidx]{amsart}

\usepackage[]{amsdip} % Options : ct, thm, couleur, draft, minimal

\pagenumbering{gobble}
\title{}

\begin{document}
\section{Complexes et suites exactes}
\subsection{Filtration}
si on a une suite d'ev $F^a$ avec des morphismes $f^a : F^a \to F^{a+1}$, de graphe $\phi^a = \Gamma(f^a) \in F^a \oplus F^{a+1}$.

On pose
\[
F = \bigoplus_{a \in \mathbb{Z}} F^a
\]
\[
\Phi = \bigoplus_{a \in \mathbb{Z}} \phi^a \quad \subseteq \bigoplus_{a \in \mathbb{Z}} \left(F^a\oplus F^{a+1}\right)
\]

Alors si $F^\bullet$ est un complexe, $\Phi$ induit une graduation $B$ sur $F$ donnée par
\[
B^k = \left(\bigoplus_{a<k} F^a\right) \oplus \phi^a
\]
Mais de façon évidente, il y a une autre graduation $A$ de $F$ : celle donnée par
\[
A^k = \bigoplus_{a \leq k} F^a \subseteq F
\]

On a alors
\[
A^k \cap B^k = ?
\]

\vfill
\newpage
\subsection{Complexe Gradué}
On peut supposer que chaque $F^a$ vient avec un graduation
\[
F^a \supset \cdots \supset G^pF^a \supset G^{p+1}F^a \supset \cdots 0
\]
qui est préservé par le complexe au sens suivant
\[
f^a\left(G^pF^a\right) \subseteq G^pF^{a+1}
\]

\subsubsection{Cas le plus simple : Un sous-complexe avec la dérivation induite}
Soit $G^\bullet \subseteq F^\bullet$ un sous-complexe stable par $f^\bullet$.

\paragraph{\textbf{Page 0}}
\begin{center}
\begin{tikzpicture}
\matrix (m) [matrix of math nodes, row sep=2em,
column sep=2.5em, text height=1.5ex, text depth=0.25ex]
{ \cdots & G^a & G^{a+1} & G^{a+2} & \cdots \\
  \cdots & F^{a-1}/G^{a-1} & F^a/G^a & F^{a+1}/G^{a+1} & \cdots \\ };
\path[->, font=\scriptsize]
(m-1-1) edge node[auto] {} (m-1-2)
(m-1-2) edge node[auto] {} (m-1-3)
(m-1-3) edge node[auto] {} (m-1-4)
(m-1-4) edge node[auto] {} (m-1-5)
(m-2-1) edge node[auto] {} (m-2-2)
(m-2-2) edge node[auto] {} (m-2-3)
(m-2-3) edge node[auto] {} (m-2-4)
(m-2-4) edge node[auto] {} (m-2-5);
\end{tikzpicture}
\end{center}

On calcule alors les $E_1^{p,q}$ comme la cohomologie des complexes suivants
\[
 F^{a-1}/G^{a-1} \longrightarrow \underset{E^{0,a}_1}{F^a/G^a} \longrightarrow F^{a+1}/G^{a+1}
\]
et
\[
G^a \longrightarrow \underset{E^{1,a}_1}{G^{a+1}} \longrightarrow G^{a+2}
\]

\paragraph{\textbf{Page 1}}
\begin{center}
\begin{tikzpicture}
\matrix (m) [matrix of math nodes, row sep=2em,
column sep=2.5em, text height=1.5ex, text depth=0.25ex]
{ \cdots & E^{1,a-1} & E^{1,a} & E^{1,a+1} & \cdots \\
  \cdots & E^{0,a-1} & E^{0,a} & E^{0,a+1} & \cdots \\ };
\path[->, font=\scriptsize]
(m-2-2) edge node[auto] {} (m-1-2)
(m-2-3) edge node[auto] {} (m-1-3)
(m-2-4) edge node[auto] {} (m-1-4);
\end{tikzpicture}
\end{center}



On calcule alors les $E_2^{p,q}$ comme la cohomologie du complexe suivant
\[
\cdots \to 0 \to 0 \to  \underset{E_2^{0,a}}{E_1^{0,a}} \to \underset{E_2^{1,a}}{E_1^{1,a}} \to 0 \to 0 \to \cdots
\]
ce qui donne en fait la suite exacte
\[
0 \to E^{0,a}_2 \to  E_1^{0,a} \to E_1^{1,a} \to E^{1,a}_2 \to 0
\]


\paragraph{\textbf{Page 2}}
\begin{center}
\begin{tikzpicture}
\matrix (m) [matrix of math nodes, row sep=2em,
column sep=2.5em, text height=1.5ex, text depth=0.25ex]
{ \cdots & E^{1,a-1} & E^{1,a} & E^{1,a+1} & \cdots \\
  \cdots & E^{0,a-1} & E^{0,a} & E^{0,a+1} & \cdots \\ };
\end{tikzpicture}
\end{center}


\subsection{Interprétation à 1 object}
La mise en commun de tous les $f^a$ donne un endomorphisme $f$ de $F$. Si $F^\bullet$ est un complexe, alors $f^2 = 0$. Soit
\[
G = \dfrac{\text{ker }f}{\text{Im }f}
\]

\begin{itemize}
\item
$f^2$ mesure l'obstruction à ce que $F^\bullet$ soit un complexe
\item
Si $F^\bullet$ est un complexe, $G$ mesure l'obstruction à ce que $F^\bullet$ soit exacte.
\end{itemize}
\end{document}