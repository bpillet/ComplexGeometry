Cohomologie

\chapter{Notations}
%\documentclass[a4paper,11pt,draft,makeidx,twocolumn]{amsart}
\usepackage[thm,couleur,draft]{amsdip}

\geometry{hmargin=3em}
\setlength{\columnsep}{2em}
\DeclareMathOperator\uu{\underline{u}}
\pagenumbering{gobble}

\begin{document}
\section{Notations}
\subsection*{Données.}
Soit $M$ une variété réelle lisse ou analytique de dimension $2n$ munie d'une structure complexe $I$ intégrable. On dispose au voisinage d'un point $O \in M$ de coordonnées réelles $x^i$ centrées en $O$. On notera $u^a = x^a + ix^{a+n}$	.

On introduit les objets suivants~:
\subsection{} Les différents faisceaux naturels, (on notera $\Gamma(U,\Ff)$ les sections de $\Ff$ sur l'ouvert $U$)~:
\begin{itemize}
\item Les faisceaux constants $\underline{\R}$, $\underline{\C}$. Dont les sections sont les fonctions localement constantes à valeurs dans $\R$, resp. $\C$.
\item Le faisceau structural (lisse) $\Cc^\infty$ ou $\Cc^\infty_M$, dont les sections sont les fonctions lisses à valeurs complexes.
\item Le faisceau structural analytique $\Cc^\omega$ ou $\Cc^\omega_M$, dont les sections sont les fonctions analytiques à valeurs complexes.
\end{itemize}
\subsection{} Le fibré tangent $T_\R M$ à $M$, ses sections lisses forment de manière naturelle un faisceau de $\R$-espaces vectoriels de dimensions $2n$ que l'on notera $\mathfrak{X}$. Une base locale des sections est donnée par la famille~:
\[
\dpp{}{x^1} , \dpp{}{x^2} , \cdots , \dpp{}{x^{2n}}
\]
On identifie les champs de vecteurs et les dérivations réelles sur l'anneau des fonctions réelles sur $M$.
\subsection{} Son complexifié $T_\C M := T_\R M \otimes_\R \C$ dont le faisceau associé est $\mathfrak{X}_\C = \mathfrak{X} \otimes_{\underline{\R}} \underline{\C}$.
Une base locale des sections est donnée par la famille précédente ou par~:
\[
\dpp{}{u^1} , \dpp{}{u^2} , \cdots , \dpp{}{u^{n}} , \dpp{}{\bar{u}^1} , \dpp{}{\bar{u}^2} , \cdots , \dpp{}{\bar{u}^{n}}
\]
On a la décomposition spectrale (fibre à fibres)
\[
T_\C M = T^{1,0}M \oplus T^{0,1}M
\]
qui s'écrit en terme de faisceaux
\begin{equation}
\mathfrak{X}_\C = \mathcal{X} \oplus \bar{\mathcal{X}}
\end{equation}

qui fait apparaître
\begin{itemize}
\item L'espace propre pour la valeur propre $i$ de l'opérateur $I$~: $T^{1,0}M$. Fibré vectoriel complexe de dimension $n$ dont les sections sont appelés \textit{champs de vecteurs $I$-holomorphes}.\\
Il s'identifie naturellement au fibré tangent réel par l'application partie réelle $T^{1,0}M~\rightarrow~T_\R M$ ; on le notera par la suite $TM$.
\item L'espace propre pour la valeur propre $-i$ de l'opérateur $I$~: $T^{0,1}M$. Fibré vectoriel complexe de dimension $n$ dont les sections sont appelés \textit{champs de vecteurs \mbox{($I$-)antiholomorphes}}.
\item Une opération $X \mapsto \bar{X}$ sur $T_\C M = T_\R M \otimes_\R \C$ qui échange $T^{1,0}M$ et $T^{0,1}M$.
\end{itemize}
\subsection{} L'espace des $1$-formes réelles $\Omega_{\R,M} := \Hom_\R(T_\R M,\R)$.
\subsection{} Son complexifié $\Omega_{\C,M} := \Hom_\C(T_\C M,\C) = \Hom_\R(T_\R M,\C)$. On a la décomposition obtenue par dualité~:
\[
\Omega_{\C,M} = \Omega^{0,1} \oplus \Omega^{1,0}
\]
\begin{itemize}
\item $\Omega^{1,0} := (T^{0,1}M)^\bot$ faisceau des formes qui s'annulent sur les $(0,1)$-vecteurs. Fibré vectoriel complexe de dimension $n$.\\
C'est également l'espace des formes propres de valeur propre $i$ pour l'operateur $I^*$.
\item $\Omega^{0,1} := (T^{1,0}M)^\bot$ faisceau des formes qui s'annulent sur les $(1,0)$-vecteurs. Fibré vectoriel complexe de dimension $n$.\\
C'est également l'espace des formes propres de valeur propre $-i$ pour l'operateur $I^*$.
\end{itemize}
\subsection{} On définit les $m$-formes à valeur complexes par~:
\[
\Omega^m := \bigwedge^m \Omega_{\C,M}
\]
C'est le faisceau des formes $m$-linéaires alternées sur $TM$ à valeurs complexes. On remarquera que $\Omega^0 = \Cc^\infty_M$ faisceau des fonctions complexes lisses.
\subsection{ \label{forme_type}} Enfin, on définit les $(p,q)$-formes de la façon suivante~:
\[
\Omega^{p,q} : = \bigwedge^p \Omega^{1,0}  \wedge \bigwedge^q \Omega^{0,1} \; \subset \; \Omega^m
\]
Si on pose $m=p+q$. C'est également le faisceau des formes $m$-linéaires alternées sur $T_\C M$ qui s'annulent sur les $m$-uplets de vecteurs $(X_1,\cdots X_{m})$ dès lors que
\begin{itemize}
\item au moins $p+1$ des $X_i$ sont de type $(1,0)$
\item ou au moins $q+1$ des $X_i$ sont de type $(0,1)$.
\end{itemize}
On a alors la décomposition
\[
\Omega^m = \bigoplus_{p+q = m} \Omega^{p,q}
\]
\subsection{} Les opérateurs $\dd : \Omega^k \rightarrow \Omega^{k+1}$ en définissant $(\dd \theta)(X)$ pour $\theta \in  \Omega^k$, $X=(X_1,X_2, \cdots , X_{k+1})$ où les $X_i \in TM$ par~:
\[
\sum_{1 \leq j \leq k} (−1)^j X_j\left(\theta\left(\check{X}^j\right)\right)+
\sum_{1 \leq j < i \leq k} (−1)^{j+i} \theta\left([X_j,X_i ], \check{\check{X}}^{j,i} \right)
\]
\begin{itemize}
\item Dans le cas $k=0$, la deuxième partie de la formule est vide, et on retrouve l’opération $f \mapsto X(f)$, ainsi $(\dd f)(X) = X(f)$.
\item Cette définition intrinsèque coïncide dans des coordonnées avec
\[
\dd \theta = \dd \left(\theta_K \dd x^K\right) = \dpp{\theta_K}{x^i}\dd x^i \wedge \dd x^K
\]
\item L’opérateur $d$ sur $ \Omega^k$ satisfait la règle de \textsc{Leibnitz} : 
\[
\dd\,(\alpha \wedge \beta) = \dd \alpha \wedge \beta +(-1)^{\deg(\alpha)} \alpha \wedge \dd \beta
\]
\end{itemize}
\subsection{} On dispose naturellement des projections
\[
\pi^{p,q} : \Omega^{p+q} \longrightarrow \Omega^{p,q}
\]
On peut dès lors définir les opérateurs $\partial$ et $\dbarre$ comme
\begin{itemize}
\item la partie de type $(p+1,q)$ de la différentielle d'une $(p,q)$-forme : $\partial = \pi^{p+1,q} \circ \dd_{\,\vert \Omega^{p,q}}$
\item la partie de type $(p,q+1)$ de la différentielle d'une $(p,q)$-forme : $\dbarre = \pi^{p,q+1} \circ \dd_{\,\vert \Omega^{p,q}}$
\end{itemize}
\subsection{} On définit $\Oo_M$ (ou $\Oo_{(M,I)}$ si il y a ambiguïté) le faisceau des fonctions holomorphes sur $M$ à valeurs complexes comme le noyau de l’opérateur $\dbarre : \Cc^\infty_M \rightarrow \Omega^1$. C'est automatiquement un sous-faisceau de $\Cc^\omega_M$ (conséquence de la formule de \textsc{Cauchy}).\\
De même on définit $\Oo_M(E)$ pour $E \rightarrow M$ fibré vectoriel holomorphe, comme le faisceau des sections holomorphes de $E$.
\subsection{\label{integrabilité}} La structure presque complexe $I$ est dite \emph{intégrable} si l'une des conditions équivalentes est satisfaite~:
\begin{enumerate}[($i$)]
\item Le tenseur de \textsc{Nijenhuis} défini par~: $N_I(X,Y) := [ I X , I Y ] - I [ X , I Y ] - I [ I X , Y ] + I^2 [ X , Y ]$ est identiquement nul.
\item Pour tout champ de vecteur complexes $X,Y$ sur $M$ satisfaisant $I X=iX$ et $I Y=iY$, le crochet de \textsc{Lie} $[X,Y]$ satisfait également $I[X,Y] = i[X,Y]$
\item L'espace tangent $I$-holomorphe $TM = T^{1,0}M \subset T_\C$  est stable par crochet de Lie. C'est-à-dire $[TM,TM] \subseteq TM$
\item Pour toute famille $(\omega^\alpha)_\alpha$ de $(1,0)_I$-formes sur $M$ de rang $n$, et pour tout $\alpha$, $\dd \omega^\alpha = \theta^\alpha_\beta \wedge \omega^\beta$ pour des $1$-formes $\theta^\alpha_\beta$.
\item Le dual de l'espace tangent $I$-holomorphe $\Omega^{1,0} \subseteq \Omega^1$ satisfait $\dd \Omega^{1,0} \subseteq \Omega^1 \wedge \Omega^{1,0}_M$.
\item $\dd = \partial + \dbarre$, ce qui signifie que la différentielle d'une $(p,q)$-forme est une somme de $(p+1,q)$ et $(p,q+1)$-formes.
\item $\dd  = \partial + \dbarre$ sur $\Omega^1_M$.
\item Le diagramme suivant commute~:
\begin{center}\begin{tikzpicture}
\matrix (m) [matrix of math nodes, row sep=2em,
column sep=2.5em, text height=1.5ex, text depth=0.25ex]
{ \Omega^{0,1} & \Omega^1 & \Omega^{1,0} \\
  \Omega^{0,2} & \Omega^2 & \Omega^{2,0} \\ };
\path[->, font=\scriptsize]
(m-1-2) edge node[auto] {$\pi^{0,1}$} (m-1-1)
(m-1-2) edge node[auto] {$\pi^{1,0}$} (m-1-3)
(m-2-2) edge node[auto] {$\pi^{0,2}$} (m-2-1)
(m-2-2) edge node[auto] {$\pi^{2,0}$} (m-2-3)
(m-1-1) edge node[auto] {$\dbarre$} (m-2-1)
(m-1-2) edge node[auto] {$d$} (m-2-2)
(m-1-3) edge node[auto] {$\partial$} (m-2-3);
\end{tikzpicture}\end{center}
Cette propriété d'intégrabilité est détaillé en appendice \autoref{preuveintegrabilite}.
\item $\partial^2 f = 0$ pour tout $f \in\Gamma(M,\Cc^\infty)$.
\end{enumerate}

\subsection*{Todo}
\begin{itemize}
\item Anti-symétrisation
\item Notation tensorielle
\item $\dd$ (définition tensorielle)
\end{itemize}

\subsection*{Formulaire.}
\begin{eqnarray*}
\dd f & = & \dpp{f}{z}\ \dd z + \dpp{f}{\bar{z}}\ \overline{\dd z} \\
\dbarre f & = & \dpp{f}{z}\ \dbarre z + \dpp{f}{\bar{z}}\ \overline{\partial z} \\
\partial f & = & \dpp{f}{z}\ \partial z + \dpp{f}{\bar{z}}\ \overline{\dbarre z}
\end{eqnarray*}
\end{document}

\section{Faisceaux classiques}

\subsection{Fonctions localement constantes}
$\underline{R}$ faisceau des fonctions localement constantes à valeur dans $R$

\subsubsection{Structurels}
Fonction, par défaut, signifie à valeur dans $\C$
\begin{itemize}
\item $C$ : continues
\item $C^k$ : $k$-fois dérivables à dérivées continue
\item $\Epsilon$ : lisses
\item $C^\omega$ : analytiques
\item $\Oo$ : holomorphes
\item $\underline{o}$ : polynômiales
\end{itemize}

Fonctions à valeur dans $\R \cup \{+\infty\}$
\begin{itemize}
\item $\Hh$ : harmoniques
\item $PSh$ : plurisousharmoniques
\end{itemize}

\subsection{formes et vecteurs}

$\mathcal{A}^p = \Epsilon^p$ : $p$-formes différentielles
$\mathcal{A}^{p,q} = \Epsilon^{p,q}$ : $p+q$-formes différentielles de type $(p,q)$ relativement à une structure presque-complexe
$\Omega^p$ : $p$-formes holomorphes
$\Omega^{p,q}$ : $p+q$ formes holomorphes de type $(p,q)$

$\mathfrak{X}$ : champs de vecteurs lisses
$\mathcal{X}$ : champs de vecteurs holomorphes

Cas particuliers
\[
\mathcal{A}^0 = \C^\infty
\]
\[
\Omega^0 = \Oo
\]

\subsubsection{Courants}
\cite{Demailly}

\subsection{Faisceau gratte-ciel}

\section{Pousser et tirer}

\subsection{!}

\subsection{-1}
exact

\subsubsection{\^*}

\subsection{\_*}
exact à gauche

\subsubsection{Higher pushforwar \$R\^qf\_*\$}

\subsection{\^!}

\subsection{\$\otimes\$}
Exact si plat

\chapter{Algebrique}

\section{Cohomologie d'un complexe}
Demailly Chap IV.

\section{Cohomologie des Faisceaux}

\subsection{Faisceau flasques et résolution flasque}
Demailly Chap. IV

\subsection{Cohomologie des faisceaux}

\subsection{Suite exacte longue}

\subsection{Suite de Mayer-Vietoris}

\section{Cech}

\section{Théorème de Leray des faisceaux acycliques}
\begin{defi}
Un recouvrement $\mathfrak{U}$ de $X$ est acyclique pour $\Ff$ si $\Ff$ n'as pas de cohomologie supérieure sur les intersections d'ouverts de $\mathfrak{U}$. C'est-à-dire~:
\[
\forall p>0, \forall k>0, \forall J \subseteq I, \vert J \vert = k \quad
\check{H}^p(U_J,\Ff) = 0
\]
\end{defi}

\begin{thm}[leray]\label{Thm_Leray}
Si $\mathfrak{U}$ est un recouvrement acyclique pour $\Ff$, alors
\begin{equation}
\forall p, \quad H^p(X,\Ff) = \check{H}^p(\mathfrak{U},\Ff)
\end{equation}
\end{thm}
morale : Coh de Cech sur un recouvrement acyclique = Coh des Faisceaux

\cite{Demailly} Chap IV
par. 5. Cech Cohomologie

\begin{cor}
Si $\mathfrak{U}$ est un recouvrement acyclique pour $\Ff$, alors
\begin{equation}
\forall p, \quad \check{H}^p(X,\Ff) = \check{H}^p(\mathfrak{U},\Ff)
\end{equation}
\end{cor}

\subsection{Lemme d'acyclicité}
Sur une variété paracompacte, les faisceaux de $C^\infty$-modules sont acycliques sur les recouvrements localement finis.

En particulier pour de tels recouvrements, les faisceaux de $C^\infty$-modules n'ont pas de cohomologie supérieure non nulle : 
\begin{equation}
\forall q>0 , \quad \check{H}^q(X,\Ff) = 0
\end{equation}

\subsection{Sur un espace paracompact,
Cech=Faisc.}
Si $X$ est paracompacte et $\Ff$ un faisceau sur $X$ alors
\begin{equation}
\forall q \geq 0, \quad \cech{H}^q(X,\Ff) = H^q(X,\Ff)
\end{equation}

\section{Isomorphisme de Leray}

\chapter{Geometrique}

\section{Betti}

\section{De Rham}
Par le lemme de Poincaré la suite
\begin{equation}\label{ExSeq_DR}
0 \rightarrow \underline{\C} \rightarrow C^\infty \rightarrow_d
\mathcal{A}^1 \rightarrow_d \mathcal{A}^2 \cdots
\end{equation}
 est exacte.

C'est une résolution injective du faisceau $\underline{C}$ (référence ?).

\subsection{Lemme de Poincaré}
\begin{lem}[Poincaré]
Soit $U$ ouvert simplement connexe de $\R^n$ et soit $\omega$ une $k$-forme $\dd$-fermée sur $U$ alors il existe $\theta$ une $k-1$-forme sur $U$ telle que $\omega = \dd theta$
\end{lem}



\section{Dolbeault}

\subsection{Lemme de Dolbeault-Poincaré}
\begin{lem}[]
Soit $U$ ouvert simplement connexe de $\C^n$ et soit $\omega$ une $(p,q)$-forme $\dbarre$-fermée sur $U$ alors il existe $\theta$ une $(p,q-1)$-forme sur $U$ telle que $\omega = \dbarre theta$
\end{lem}

\begin{lem}[Griffiths-Harris]
Soit $\Delta$ polydisque de $\C^n$, alors
\begin{equation}
H^{p,q}_{Dol}(\Delta,\underline{C}) = 0 \quad \text{ pour } q \geq 1
\end{equation}
\end{lem}

\section{Bott-Chern}
\[
\dfrac{\ker \partial \cap \ker \bar\partial}{\Im \partial\bar\partial}
\]

\section{Aeppli}
\[
\dfrac{\ker \partial\bar\partial}{\Im \partial + \Im\bar\partial}
\]

\chapter{Relations}

\section{Vanishing Thm}

\subsection{Kodaira}

\section{Dualités}

\subsection{Poincaré}

\subsection{Serres}

\section{Künneth formula}
[Formule de Künneth pour l'homologie, \cite{GH} I.4 p. 58]
\begin{equation}\label{Kunneth}
H_{k}(X\times Y,\Q) \simeq \bigoplus_{p+q=k} H_p(X,\Q) \otimes_\C H_q(Y,\Q)
\end{equation}

Si $X$, $Y$ variétés complexes dont l'une au moins est compacte. $\Ff$ et $\Gg$ des faisceaux de $\C$-ev.
[Formule de Künneth pour la cohomologie, \cite{Demailly} Chap IV par. 15 p.278]
\begin{equation}\label{CoKunneth}
H_{k}(X \times Y,\Ff \sqtimes \Gg) \simeq \bigoplus_{p+q=k} H_p(X,\Ff) \otimes_\C H_q(Y,\Gg)
\end{equation}

\chapter{Théorie de Hodge}

\section{Théorème de Hodge}
Métrique hermitienne
->
Notion de forme harmonique (et dimension finie des espaces de formes harmoniques)
->
Représentant harmonique des classes de Cohomologies

\subsection{Le cas Kahler}

\chapter{Non-Classé}

\section{Interpretation du H\^1 en terme d'extensions}
$H^1(X,\Ff)$ classifie les suites exactes
\begin{equation}
0 \rightarrow \Ff \rightarrow \Gg \rightarrow \Oo_X \rightarrow 0
\end{equation}

Ah bon ?

Dans le cas $X$ compact, $H^0(X,\Oo_X) = \Cc$ et donc la suite exacte longue de cohomologie associée à une telle extension $\Gg$ nous donne
\begin{equation}
\cdots \rightarrow H^0(X,\Oo_X) \rightarrow H^1(X,\Ff) \rightarrow \cdots
\end{equation}
et donc un élément de $H^1(X,\Ff)$ obtenue comme l'image de $1 \in \Cc \cong H^0(X,\Oo_X)$.

Réciproquement ?


\section{Propriétés des faisceaux}

\subsection{Flasque (flabby)}
Toute section locale peut être étendue en une section globale.

Ex: Si $X$ n'est pas discret, même $C_X$ n'est pas flasque.

\subsubsection{Les faisceaux flasques sont acycliques}

\subsection{Mou (Soft)}
Toute section sur un fermé $S$ de $X$ peut-être étendue en une section globale

Ex : $C_X$ (théorème de Tietze-Urysohn) et $C_X^\infty$

Si $R$ est un faisceau d'anneaux doux, alors tous les faisceaux de $R$-modules sont doux.

\subsection{Acyclique}
\begin{defi}[Faisceau acyclique \label{acyclique}\index{faisceau acyclique}]
Un faisceau acyclique $F$ sur  $X$ est un faisceau dont tous les groupes de cohomologie supérieure sont nuls.
\end{defi}

\subsubsection{\$C\^\infty\$-modules}
Les faisceaux de $C^\infty$-modules sont acycliques

Bilan : 
\begin{prop}
Toute suite exacte de faisceaux de $\Cc^\infty$-modules est $C^\infty$ scindée.
\end{prop}

\[
0 \rightarrow E \rightarrow F \rightarrow G \rightarrow 0
\]

Alors en prenant le produit tensoriel au dessus de $\Cc^\infty$ par $G^*$, on obtient la suite exacte~:

\[
0 \rightarrow \Hom(G,E) \rightarrow \Hom(G,F) \rightarrow \End(G) \rightarrow \Ext^1(G,E) \rightarrow \cdots
\]
Or $\Hom(G,E)$ est un faisceau acyclique donc n'a pas de cohomologie supérieure.

Ainsi il existe un antécédent dans $\Hom(G,F)$ à l'identité dans $\Hom(G,G)$, c'est-à-dire une section de $F \rightarrow G$. Donc la suite est scindée.

\subsection{Fin (Fine)}

\section{Suite spectrale de Frölisher}

\subsection{Le cas Kahler}
Dégénère en page 1

\chapter{Programme}
