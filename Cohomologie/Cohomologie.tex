% Created 2016-06-22 mer. 10:34
\documentclass[12pt,makeidx]{amsart}
                 \usepackage[couleur,draft]{/home/basile/Git/Latex/Headfiles/amsdip}
                 \usepackage[utf8x]{inputenc}
                 \usepackage[T1]{fontenc}
                          \newtheorem{prop}{Proposition}
\newtheorem{cor}{Corollaire}
\newtheorem{lem}{Lemme}
\newtheorem{thm}{Théoreme}
\newtheorem{theorem}{Theorem}
\newtheorem{defi}{Définition}
\DeclareMathOperator{\Ext}{Ext}

\providecommand{\alert}[1]{\textbf{#1}}

\title{Cohomologie}
\author{}
\date{}
\hypersetup{
  pdfkeywords={},
  pdfsubject={},
  pdfcreator={Emacs Org-mode version 7.9.3f}}

\begin{document}

\maketitle

\setcounter{tocdepth}{3}
\tableofcontents
\vspace*{1cm}

Retour au \href{file:///home/basile/Git/Geometrie Complexe/Programme.org}{fichier général}                                          :ATTACH:

\section{Cohomologie}
\label{sec-1}
\subsection{Notations}
\label{sec-1-1}
\subsubsection{Faisceaux classiques}
\label{sec-1-1-1}
\begin{itemize}

\item Fonctions localement constantes\\
\label{sec-1-1-1-1}%
$\underline{R}$ faisceau des fonctions localement constantes à valeur dans $R$
\begin{itemize}

\item Structurels\\
\label{sec-1-1-1-1-1}%
Fonction, par défaut, signifie à valeur dans $\C$

\begin{description}
\item[$\Cc$] continues
\item[$\Cc^k$] $k$-fois dérivables à dérivées continue
\item[$\Ee$] lisses
\item[$\Cc^\omega$] analytiques
\item[$\Oo$] holomorphes
\item[$\underline{o}$] polynômiales
\end{description}


Fonctions à valeur dans $\R \cup \{+\infty\}$

\begin{description}
\item[$\Hh$] harmoniques
\item[$PSh$] plurisousharmoniques
\end{description}

\end{itemize} % ends low level

\item formes et vecteurs\\
\label{sec-1-1-1-2}%
\begin{description}
\item[$\mathcal{A}^p = \Ee^p$] $p$-formes différentielles
\item[$\mathcal{A}^{p,q} = \Ee^{p,q}$] $p+q$-formes différentielles de type $(p,q)$ relativement à une structure presque-complexe
\item[$\Omega^p$] $p$-formes holomorphes
\item[$\Omega^{p,q}$] $p+q$ formes holomorphes de type $(p,q)$
\item[$\mathfrak{X}$] champs de vecteurs lisses
\item[$\mathcal{X}$] champs de vecteurs holomorphes
\end{description}

Cas particuliers
\[
\mathcal{A}^0 = \Cc^\infty
\]
\[
\Omega^0 = \Oo
\]
\begin{itemize}

\item Courants\\
\label{sec-1-1-1-2-1}%
\cite{Demailly}
\end{itemize} % ends low level

\item Faisceau gratte-ciel
\label{sec-1-1-1-3}%
\end{itemize} % ends low level
\subsubsection{Pousser et tirer}
\label{sec-1-1-2}

\cite{Virk}
\begin{itemize}

\item $f_!$
\label{sec-1-1-2-1}%

\item $f^{-1}$\\
\label{sec-1-1-2-2}%
exact
\begin{itemize}

\item $f^*$
\label{sec-1-1-2-2-1}%
\end{itemize} % ends low level

\item $f_*$\\
\label{sec-1-1-2-3}%
exact à gauche
\begin{itemize}

\item Higher pushforwar $R^qf_*$
\label{sec-1-1-2-3-1}%
\end{itemize} % ends low level

\item $f^!$
\label{sec-1-1-2-4}%

\item $f \otimes$\\
\label{sec-1-1-2-5}%
Exact si plat
\end{itemize} % ends low level
\subsection{Algebrique}
\label{sec-1-2}
\subsubsection{Cohomologie d'un complexe}
\label{sec-1-2-1}

\cite{Demailly} Chap IV.
\cite{Voisin}

Dans une catégorie abélienne $\Aa$, un complexe est la donnée du couple $(M^\bullet, d^\bullet)$ tel que $d^i : M^i \to M^{i+1}$ et $d^{i+1} \circ d^i = 0$ pour tout $i \in \Z$.

La cohomologie en degré $i$ d'un complexe $(M,d)$ est l'objet de la catégorie $\Aa$
\[
H^i(M^\bullet) := \text{Coker}\left( d^{i-1}: M^{i-1} \to \ker d^i\right)
\]
\begin{itemize}

\item Hypercohomologie d'un complexe de faisceaux\\
\label{sec-1-2-1-1}%
\cite{Voisin}

Soit $\Ff^\bullet$ un complexe de faisceaux de groupes abéliens sur un espace topologique $X$. Alors on définit \textit{l'hypercohomologie} du complexe comme étant
\[
\Ha^{k}(X,\Ff^\bullet) := R^k\Gamma(\Ff^\bullet)
\]

Si $\Gg^\bullet$ est un complexe de faisceaux $\Gamma$-acycliques et quasi-isomorphe à $\Ff^\bullet$ alors
les $G^\bullet = \Gg^\bullet(X)$ forment un complexe de groupes abéliens
et on a
\[
\Ha^k(X,\Ff^\bullet) = H^k(G^\bullet)
\]
\end{itemize} % ends low level
\subsubsection{Cohomologie des Faisceaux}
\label{sec-1-2-2}
\begin{itemize}

\item Faisceau flasques et résolution flasque\\
\label{sec-1-2-2-1}%
\cite{Demailly} Chap. IV


\cite{Voisin}
Un complexe $(M^i,d^i)$ pour $i\geq 0$ est une \textit{résolution}  de $N$ si 
\begin{itemize}
\item $M^\bullet$ est exacte en $M^i$ pour $i>0$
\item on a une suite exacte
  \[
  0 \to N \to M^0 \to M^1
  \]
\end{itemize}


\item Cohomologie des faisceaux
\label{sec-1-2-2-2}%

\item Suite exacte longue
\label{sec-1-2-2-3}%

\item Suite de Mayer-Vietoris
\label{sec-1-2-2-4}%
\end{itemize} % ends low level
\subsubsection{Cech}
\label{sec-1-2-3}
\subsubsection{Théorème de Leray des faisceaux acycliques}
\label{sec-1-2-4}

\begin{defi}
Un recouvrement $\mathfrak{U}$ de $X$ est acyclique pour $\Ff$ si $\Ff$ n'as pas de cohomologie supérieure sur les intersections d'ouverts de $\mathfrak{U}$. C'est-à-dire~:
\[
\forall p>0, \forall k>0, \forall J \subseteq I, \vert J \vert = k \quad
\check{H}^p(U_J,\Ff) = 0
\]
\end{defi}

\begin{thm}[leray]\label{Thm_Leray}
Si $\mathfrak{U}$ est un recouvrement acyclique pour $\Ff$, alors
\begin{equation}
\forall p, \quad H^p(X,\Ff) = \check{H}^p(\mathfrak{U},\Ff)
\end{equation}
\end{thm}
morale : Coh de Cech sur un recouvrement acyclique = Coh des Faisceaux

\cite{Demailly} Chap IV
par. 5. Cech Cohomologie

\begin{cor}
Si $\mathfrak{U}$ est un recouvrement acyclique pour $\Ff$, alors
\begin{equation}
\forall p, \quad \check{H}^p(X,\Ff) = \check{H}^p(\mathfrak{U},\Ff)
\end{equation}
\end{cor}
\begin{itemize}

\item Lemme d'acyclicité\\
\label{sec-1-2-4-1}%
Sur une variété paracompacte, les faisceaux de $C^\infty$-modules sont acycliques sur les recouvrements localement finis.

En particulier pour de tels recouvrements, les faisceaux de $C^\infty$-modules n'ont pas de cohomologie supérieure non nulle : 
\begin{equation}
\forall q>0 , \quad \check{H}^q(X,\Ff) = 0
\end{equation}

\item Sur un espace paracompact,\\
\label{sec-1-2-4-2}%
Cech=Faisc.
Si $X$ est paracompacte et $\Ff$ un faisceau sur $X$ alors
\begin{equation}
\forall q \geq 0, \quad \check{H}^q(X,\Ff) = H^q(X,\Ff)
\end{equation}
\end{itemize} % ends low level
\subsubsection{Isomorphisme de Leray}
\label{sec-1-2-5}
\subsection{Geometrique}
\label{sec-1-3}
\subsubsection{Betti}
\label{sec-1-3-1}
\subsubsection{De Rham}
\label{sec-1-3-2}

Par le lemme de Poincaré le complexe
\begin{equation}\label{ExSeq_DR}
0 \rightarrow  C^\infty \rightarrow_d
\mathcal{A}^1 \rightarrow_d \mathcal{A}^2 \cdots
\end{equation}
 est exacte en $A^k$ pour tout $k>0$.

C'est une résolution du faisceau $\underline{C}$
\cite{Voisin}.
\begin{itemize}

\item Lemme de Poincaré \textbf{:Poincare:}
\label{sec-1-3-2-1}%
\begin{lem}[Poincaré\label{LemmePoincare}]
Soit $U$ ouvert simplement connexe de $\R^n$ et soit $\omega$ une $k$-forme $\dd$-fermée sur $U$ alors il existe $\theta$ une $k-1$-forme sur $U$ telle que $\omega = \dd theta$
\end{lem}


\end{itemize} % ends low level
\subsubsection{Dolbeault}
\label{sec-1-3-3}
\begin{itemize}

\item Lemme de Dolbeault-Poincaré \textbf{:Poincare:}
\label{sec-1-3-3-1}%
\begin{lem}[]
Soit $U$ ouvert simplement connexe de $\C^n$ et soit $\omega$ une $(p,q)$-forme $\dbarre$-fermée sur $U$ alors il existe $\theta$ une $(p,q-1)$-forme sur $U$ telle que $\omega = \dbarre theta$
\end{lem}

\begin{lem}[\cite{Griffiths-Harris}]
Soit $\Delta$ polydisque de $\C^n$, alors
\begin{equation}
H^{p,q}_{Dol}(\Delta,\underline{C}) = 0 \quad \text{ pour } q \geq 1
\end{equation}
\end{lem}
\end{itemize} % ends low level
\subsubsection{Bott-Chern}
\label{sec-1-3-4}

\[
\dfrac{\ker \partial \cap \ker \bar\partial}{\Im \partial\bar\partial}
\]
\subsubsection{Aeppli}
\label{sec-1-3-5}

\[
\dfrac{\ker \partial\bar\partial}{\Im \partial + \Im\bar\partial}
\]
\subsection{Relations}
\label{sec-1-4}
\subsubsection{Vanishing Thm}
\label{sec-1-4-1}
\begin{itemize}

\item Grothendieck\\
\label{sec-1-4-1-1}%
Pas de cohomologie des faisceaux en degré supérieur à la dimension. Dans tout le cadre le plus général possible.
\begin{theorem}[Grothendieck vanishing]
Let $X$ be a noetherian topological space.

For all $i > dim X$ and all sheaves of abelian groups $\Ff$ on $X$, we have $H^i(X, \Ff ) = 0$.
\end{theorem}

\item Kodaira
\label{sec-1-4-1-2}%
\begin{theorem}[Kodaira Vanishing]
$M$ is a compact Kähler manifold of complex dimension $n$, $L$ any holomorphic line bundle on $M$ that is positive (ample), and $K_M$ is the canonical line bundle, then
\[H^q(M,K_M \otimes L\]
for $q>0$.
\end{theorem}
\begin{itemize}

\item Kodaira$_{\mathrm{Nakano}}$
\label{sec-1-4-1-2-1}%
\end{itemize} % ends low level
\end{itemize} % ends low level
\subsubsection{Dualités}
\label{sec-1-4-2}
\begin{itemize}

\item Poincaré\\
\label{sec-1-4-2-1}%
Variété compacte sans bord, orientée de dimension $n$, alors
\[
H^k(M,\C) = H^{n-k}(M,\C)^*
\]

\item Serres
\label{sec-1-4-2-2}%
\end{itemize} % ends low level
\subsubsection{Künneth formula}
\label{sec-1-4-3}

[Formule de Künneth pour l'homologie, \cite{GH} I.4 p. 58]
\begin{equation}\label{Kunneth}
H_{k}(X\times Y,\Q) \simeq \bigoplus_{p+q=k} H_p(X,\Q) \otimes_\C H_q(Y,\Q)
\end{equation}

Si $X$, $Y$ variétés complexes dont l'une au moins est compacte. $\Ff$ et $\Gg$ des faisceaux cohérents sur $X$ et $Y$ respectivement.
[Formule de Künneth pour la cohomologie, \cite{Demailly} Chap IV par. 15 et Chap IX par. 5.B, 5.23]
\begin{equation}\label{CoKunneth}
H_{k}(X \times Y,\Ff \boxtimes \Gg) \simeq \bigoplus_{p+q=k} H_p(X,\Ff) \otimes_\C H_q(Y,\Gg)
\end{equation}
\begin{itemize}

\item Autre référence \href{http://mathoverflow.net/questions/34673/kunneth-formula-for-sheaf-cohomology-of-varieties}{http://mathoverflow.net/questions/34673/kunneth-formula-for-sheaf-cohomology-of-varieties}
\label{sec-1-4-3-1}%

\end{itemize} % ends low level
\subsection{Théorie de Hodge}
\label{sec-1-5}
\subsubsection{Théorème de Hodge}
\label{sec-1-5-1}

Métrique hermitienne
->
Notion de forme harmonique (et dimension finie des espaces de formes harmoniques)
->
Représentant harmonique des classes de Cohomologies
\begin{itemize}

\item Le cas Kahler
\label{sec-1-5-1-1}%
\end{itemize} % ends low level
\subsection{Non-Classé}
\label{sec-1-6}
\subsubsection{Hirzebruch-Riemann-Roch \textbf{:HRR:GRR:Thm:}}
\label{sec-1-6-1}

Soit $E$ un fibré vectoriel holomorphe sur $X$ une variété complexe compacte
\begin{equation}\label{HRR}
\chi(X,E) = \int_X \text{Ch}(E)\text{Td}(X)
\end{equation}
\begin{itemize}

\item Grothendieck-Hirzebuch-Riemann-Roch\\
\label{sec-1-6-1-1}%
Version relative où $X \to \star$ est remplacé par $f: X \to Y$ propre entre des schémas quasi-projectifs lisses. Et $E$ remplacé par un complexe borné de faisceaux.
\end{itemize} % ends low level
\subsubsection{Interpretation du H$^1$ en terme d'extensions}
\label{sec-1-6-2}

$H^1(X,\Ff)$ classifie les suites exactes
\begin{equation}
0 \rightarrow \Ff \rightarrow \Gg \rightarrow \Oo_X \rightarrow 0
\end{equation}

Ah bon ?

Dans le cas $X$ compact, $H^0(X,\Oo_X) = \Cc$ et donc la suite exacte longue de cohomologie associée à une telle extension $\Gg$ nous donne
\begin{equation}
\cdots \rightarrow H^0(X,\Oo_X) \rightarrow H^1(X,\Ff) \rightarrow \cdots
\end{equation}
et donc un élément de $H^1(X,\Ff)$ obtenue comme l'image de $1 \in \Cc \cong H^0(X,\Oo_X)$.

Réciproquement ?
\subsubsection{Propriétés des faisceaux}
\label{sec-1-6-3}
\begin{itemize}

\item Flasque (flabby)\\
\label{sec-1-6-3-1}%
Toute section locale peut être étendue en une section globale.

Ex: Si $X$ n'est pas discret, même $C_X$ n'est pas flasque.
\begin{itemize}

\item Les faisceaux flasques sont acycliques
\label{sec-1-6-3-1-1}%
\end{itemize} % ends low level

\item Mou (Soft)\\
\label{sec-1-6-3-2}%
Toute section sur un fermé $S$ de $X$ peut-être étendue en une section globale

Ex : $C_X$ (théorème de Tietze-Urysohn) et $C_X^\infty$

Si $R$ est un faisceau d'anneaux doux, alors tous les faisceaux de $R$-modules sont doux.

\item Acyclique
\label{sec-1-6-3-3}%
\begin{defi}[Faisceau acyclique \label{acyclique}\index{faisceau acyclique}]
Un faisceau acyclique $F$ sur  $X$ est un faisceau dont tous les groupes de cohomologie supérieure sont nuls.
\end{defi}
\begin{itemize}

\item $C^\infty$-modules\\
\label{sec-1-6-3-3-1}%
Les faisceaux de $C^\infty$-modules sont acycliques

Bilan : 
\begin{prop}
Toute suite exacte de faisceaux de $\Cc^\infty$-modules est $C^\infty$ scindée.
\end{prop}

\[
0 \rightarrow E \rightarrow F \rightarrow G \rightarrow 0
\]

Alors en prenant le produit tensoriel au dessus de $\Cc^\infty$ par $G^*$, on obtient la suite exacte\~{}:

\[
0 \rightarrow \Hom(G,E) \rightarrow \Hom(G,F) \rightarrow \End(G) \rightarrow \Ext^1(G,E) \rightarrow \cdots
\]
Or $\Hom(G,E)$ est un faisceau acyclique donc n'a pas de cohomologie supérieure.

Ainsi il existe un antécédent dans $\Hom(G,F)$ à l'identité dans $\Hom(G,G)$, c'est-à-dire une section de $F \rightarrow G$. Donc la suite est scindée.
\end{itemize} % ends low level

\item Fin (Fine)
\label{sec-1-6-3-4}%
\end{itemize} % ends low level
\subsubsection{Suite spectrale de Frölisher}
\label{sec-1-6-4}
\begin{itemize}

\item Le cas Kahler\\
\label{sec-1-6-4-1}%
Dégénère en page 1
\end{itemize} % ends low level

\end{document}
