\documentclass[a4paper]{book}
\usepackage[thm]{dipneuste}

\begin{document}
\chapter{Fonctions holomorphes à plusieurs variables}
\section{Rappels : formes différentielles et champs de vecteurs sur $\R^{2n}$ et $\C^n$}
On note $x^1,y^1,x^2,y^2,\cdots,x^n,y^n$ les projection de coordonnées : $\R^{2n} \rightarrow \R$. et $z^1,\cdots z^n$ celles sur $\C^n$.
On identifie $\R^{2n} \cong \C^n$ ne posant $z^k = x^k + i y^k$.
On réalise l'identification suivante~:
\[
\R^{2n} = (\R^2) \oplus (\R^2) \oplus \cdots \oplus (\R^2) = \C \oplus \C \oplus \cdots \oplus \C = \C^n
\]
\section{Formule de \textsc{Cauchy} en une variable}
\begin{thm}
Soit $K \subset \C$ un compact connexe avec un bord de classe $\Cc^1$ et $f \in \Cc^1\overline{(K,}\C)$ (c'est-à-dire il existe un voisinage ouvert $\Omega$ de $K$ tel que $f$ est de classe $\Cc^1$ sur $\Omega$ vu comme un ouvert de $\R^2$)

Alors :
\[
\forall w \in \overset{\circ}{K}  \qquad f(w) =\dfrac{1}{2i\pi} \int_{\partial K} \dfrac{f(z) \dd z}{z-w} + 
\int_K \dpp{f}{\bar{z}} \dfrac{\dd z \wedge \dd \bar{z}}{z-w}
\]
\end{thm}
\section{Formule de \textsc{Cauchy} en plusieurs variables}

\end{document}